\documentclass[11pt,a4paper,numbers=endperiod]{scrartcl}

\usepackage[ngerman]{babel}			
\usepackage[utf8]{inputenc}
\usepackage{geometry}
\usepackage[T1]{fontenc}
\usepackage{lmodern}
\usepackage{stmaryrd}
\usepackage{ulem}
\usepackage{setspace}
\usepackage[T1]{fontenc}
\usepackage{mathptmx}
\usepackage{titlesec}
\usepackage{enumitem}
\usepackage{bbm}
\usepackage{sectsty}
\sectionfont{\clearpage}

\usepackage{graphicx}
\usepackage{hyperref}
\usepackage[all]{xy}
\usepackage{amsmath,listings}
\usepackage{bigdelim}
\usepackage{arydshln}
\usepackage{lscape}
\usepackage{amsmath}
\usepackage{amssymb}			
\usepackage{amsfonts}
\usepackage{amsthm}
\usepackage{mathtools}
\usepackage{pstricks,pst-plot}
\usepackage{pst-node}
\usepackage{pstricks,pst-plot,pst-node}
\SpecialCoor
\usepackage{nicefrac}
\usepackage{tikz, forest}
\usetikzlibrary{matrix}
\usetikzlibrary{fit}
\usetikzlibrary{backgrounds}

\usetikzlibrary{shapes}
\usetikzlibrary{arrows.meta}
\graphicspath{{C:/Users/Nikolaus/Pictures/}} 

\setkomafont{sectioning}{\rmfamily\bfseries} 
\setlength\parindent{0pt}

\forestset{
	.style={
		for tree={
			base=bottom,
			child anchor=north,
			align=center,
			s sep+=1cm,
			straight edge/.style={
				edge path={\noexpand\path[\forestoption{edge},thick,-{Latex}] 
					(!u.parent anchor) -- (.child anchor);}
			},
			if n children={0}
			{tier=word, draw, thick, rectangle}
			{draw, diamond, thick, aspect=2},
			if n=1{%
				edge path={\noexpand\path[\forestoption{edge},thick,-{Latex}] 
					(!u.parent anchor) -| (.child anchor) node[pos=.2, above] {Y};}
			}{
				edge path={\noexpand\path[\forestoption{edge},thick,-{Latex}] 
					(!u.parent anchor) -| (.child anchor) node[pos=.2, above] {N};}
			}
		}
	}
}


\newcommand{\id}{\hspace*{4mm}}
\newcommand{\dif}{\mathrm{d}}
\newcommand{\logit}{\text{logit}}
\newcommand{\tit}[1]{\begin{large} \underline{\text{#1}}\end{large}}
\titleformat*{\paragraph}{\large\bfseries}




\def\QQ{{\mathbb Q}}
\def\CC{{\mathbb C}}
\def\RR{{\mathbb R}}
\def\NN{{\mathbb N}}
\def\ZZ{{\mathbb Z}}
\def\PP{{\mathbb P}}
\def\FF{{\mathbb F}}


\def\Namen{} % Namen eintragen
\def\Datum{} % Datum eintragen
\def\Titel{} % Vortragstitel eintragen
% Die Titelzeilen werden aus diesen Angaben automatisch erzeugt.
\begin{document}
	\Namen \hfill \Datum\par
	\vspace{0.25\baselineskip}
	\hrule
	\vspace{\baselineskip}
	\begin{center}
		{\LARGE\textbf{Introduction to Statistical Machine Learning}}\par
		\vspace{0.25\baselineskip}
		{\large\textsc{Uni Heidelberg}}
	\end{center}
	\hrule 
	
	\vspace{\baselineskip}
	\vspace{\baselineskip}
	\vspace{\baselineskip}
	\vspace{\baselineskip}
	\vspace{\baselineskip}
	\vspace{\baselineskip}
	\vspace{\baselineskip}
	\vspace{\baselineskip}
	\vspace{\baselineskip}
	\vspace{\baselineskip}
	
	
	
	
	\begin{center}
		{\large\itshape Erstellt von: Nikolaus Schäfer}
	\end{center}
	
	\vspace{\baselineskip}
	\vspace{\baselineskip}
	\vspace{\baselineskip}
	\vspace{\baselineskip}
	\vspace{\baselineskip}
	\vspace{\baselineskip}
	\vspace{\baselineskip}
	\vspace{\baselineskip}
	\vspace{\baselineskip}
	\vspace{\baselineskip}
	
	\begin{center}
		{\large Dozent: Junprof. Dr. F. Krüger}
	\end{center}
	\vspace{\baselineskip}
	\vspace{\baselineskip}
	\vspace{\baselineskip}
	\vspace{\baselineskip}
	\vspace{\baselineskip}
	\vspace{\baselineskip}
	\vspace{\baselineskip}
	\vspace{\baselineskip}
	\vspace{\baselineskip}
	\vspace{\baselineskip}
	
	\begin{center}
		{\large \text{\today}}
	\end{center}
	
	\newpage
	\vspace{0.125\baselineskip}
	\tableofcontents %Inhaltsverzeichnis (wird automatisch erzeugt
	\newpage
	\begin{center}
		\LARGE \textbf{Editor's Note}
	\end{center}
	This script is my personal transcript of the lecture ''Introduction to Statistical Machine Learning'' by Junprof. Dr. Fabian Krüger in the summer semester 2018. As it is my personal transcript, I cannot guarantee that it is mistakefree. If you find any mistakes (e.g. typos etc.) please feel free to edit them. Furthermore, I did not maintain the same format throughout the semester, and now I have too little time to edit everything. However, I think the document still has a somewhat clear structure. If you want to edit the format a little, color or bold some words or sentences to make it more readable and highlight the important stuff, again please feel free to do that.\\
	Lastly I hope that it is of some use to the course. Enjoy and study hard!\\
	Niko
	\newpage
	\section{Lecture April 18th}
	\vspace{0.2\baselineskip}
	
	\onehalfspacing

\paragraph{What is Statistical Learning? (2.1)}
$ $\\

\tit{Example:} Advertising Data
	\begin{itemize}[label={--}]
		\item Sales of a product in 200 different markets
		\item Budget for advertising in each maket for three channels: TV, radio, newspaper.
		\item Question: Which advertising channel increaes sales (by how much?)
	\end{itemize}
In this example, $\underbrace{\text{sales}}_{Y}$ is the \textbf{output variable} (Y), advertising budgets are \textbf{input variables} ($X_1$: TV Budget, $X_2$: radio budget, $X_3$: newspaper budget) 


\underline{Synonomys}: input variable = predicators, regressors, features, independent variables\\
\hspace*{17.2mm} output variable = response variable, predictand, dependent variable

\underline{More generally}: Response $Y$ (quantitative, i.e. continuous number), $p$ predicators $X_1, \ldots, X_p$.\\
\id Aim: Study relationship betwenn $Y$ and $X = (X_1, \ldots, X_p)$.
\begin{align*}
	\textbf{Assuption: } Y = f(X) + \epsilon
\end{align*} 
when
\begin{itemize}[label={--}]
	\item $f$ is a fixed but unknown function of $X_1, \ldots, X_p$
	\item $\epsilon$ is a random error term, independent of $X$, with $E(\epsilon) = 0$
\end{itemize}
$\Rightarrow f$ represents systematic information that $X$ provides about $Y$

\paragraph{ Why estimate $f$? (2.1.1)}
$ $\\

Two motivations: Prediction and inference\\

\textbf{\tit{Prediction:}}\\
Predict $Y$ using $X$, i.e. $\hat{Y} = \hat{f}(X)$ 

\id Main goal: Accurate prediction, i.e. $\hat{Y}$ as close as possible to $\hat{f}(X)$ (see Section 2.2 below)\\

\tit{Example}: Predict next quarters GDP growth rate, $Y$, based on $X = (X_1, \ldots, X_p)$.\\
$X$ might include other other economic variables like current (this quarter's) GDP growth rate, inflation rate, interest rates, survey forecasts (e.g. survey of Professional Forecastors conducted by ECB).\\

Two sources of error: $\underset{\text{(RE)}}{\textbf{Reducible error}}$ and $\underset{\text{(IE)}}{\textbf{irreducible error}}$\\

\underline{RE}: Arises because $\hat{f}$ is an imperfect estimate of $f$ $\Rightarrow$ Can be reduced by using the most appropriate statistical learning technique to estimate $f$\\

\underline{IE}: Arises because $\epsilon$ is there and, by definition, cannot be predicted using $X$. In other words, $IE$ would still be there if we had a perfect estimate of $f$ (such that $\hat{f}(X) = f(X)$).\\

Main reason for existence of $\epsilon$: Do not/can not measure all variables $X$ that are relevant for $Y$. (for legal, technical, or ethical reasons)

\tit{Example:}
\begin{eqnarray*}
 Y &=& \text{Price of a house}\\
 X &=& \text{information on location, number of rooms, size of the property}
\end{eqnarray*}
Possible factors that go into $\epsilon$: Neighbors, urgent sales, psychological factors .\\

\section{Lecture April 19th}

\begin{eqnarray*}
Y &=& f(X) + \epsilon \hspace*{5mm} \text{model}\\
\hat{Y} &=& \hat{f}(X) \hspace{7mm} \text{forecast}
\end{eqnarray*}

\underline{Reducible error}: Results from having poor estimate $\hat{f}$ of $f$\\
\underline{Irreducible error}: Factors in $\epsilon$ are unobservable.\\

\tit{Example:} Predict children's income form parents' income.
\begin{align*}
	\underbrace{Y}_{\text{child income}} = f(\underbrace{X}_{\text{information on parent income}}) + \overbrace{\epsilon}^{\text{variables that are unrelated to parents income}}\\
\end{align*}

\begin{eqnarray*}
E\{(Y-\hat{Y})^2\} &=& E\{(f(X) + \epsilon - \hat{f}(X))^2\} = E\{(f(X) - \hat{f}(X)+ \epsilon)^2\} \hspace*{4mm}\text{Note: Assume that X is} \\
\hspace*{20.5 mm} &=& \underbrace{(f(X)- \hat{f}(X))^2}_{\text{reducible error}} + \underbrace{E(\epsilon^2)}_{V(\epsilon)} + (f(X) - \hat{f}(X)) \underbrace{E(\epsilon)}_{= 0} \hspace*{10mm} \text{fixed (no random variable)}
\end{eqnarray*}
This class: Focus on techniques for estimating $f$, with the aim to minimize the reducible error.\\

\textbf{\tit{Inference}}

Main question: How is $Y$ affected if $X_1, \ldots, X_p$ change?\\
Need to know/estimate precise structure of $f$ to understand the relation between $X$ and $Y$.\\

Specific inference questions:
\begin{itemize}[label={--}]
	\item Which predictors are associated with $Y$?
	\item What's the specific form of the relation between $Y$ and $X_1$ (e.g. linear, quadratic, logarithmic, positive, negative,...)
\end{itemize}

\tit{Example}: Clinical trial where a random subgroup of patients gets a drug (treatment group), all others receive a placebo (control group). Based on such data, one can study the relation between a patient's health and whether the patient got the drug (yes/no).\\

Examples on prediction vs. inference (Einav + Levin, 2014)

\underline{Prediction}: 
\begin{itemize}[label={--}]
	\item Predict children's wealth from parents' wealth
	\item Predict whether a user will click on an add
\end{itemize}

\underline{Inference}:
\begin{itemize}[label={--}]
	\item Ask whether taking online classes increases wage
	\item Evaluate long-run effects of teachers
\end{itemize}

In some examples, both prediction and inference may be of interest. E.g. $Y$ = value of house, $X$ = features\\

\indent \underline{Prediction}: E.g. realtor/salesperson want to predict value of house as it is.\\
\indent \underline{Inference}: Find variable that affect house price (in order to maximize value given resources)

\paragraph{How do we estimate $f$? (2.1.2)}
$ $\\ 

The data set, that is used to estimate $f$ is called \textbf{training data}, given by $n$ observations. Let $y_i$ denote the oberserved response for obersvation $i = 1, \ldots, n$ and $x_{ij}$ the value for observation $i$ and predictor $j = 1, \ldots p$.\\
We denote $x_i= (x_{i1}, \ldots, x_{ip})^T$, so that our training data set consists of $\{(x_1, y_1), \ldots, (x_n, y_n)\}$\\

Based on the training data, we aim to estimate $f$ such that $\hat{f}(X) \approx Y$ for any observation $(X, Y)$. Two categories of estimation: Parametric and nonparametric.\\

\textbf{\tit{Parametric methods}}  

Two steps:
\begin{enumerate}
	\item Make an assumption about the functional form of $f$, e.g. the linear model assumes that $Y$ is a linear function of $X$: $f(X) = \beta_0 + \beta_1X_1 + \ldots \beta_p X_p$
	\item After the form  of $f$ has been selected: Use training data to estimate (or fit) the model. E.g in the linear model we would estimate the coefficients $(\beta_0, \ldots, \beta_p)$ using ordinary least squares. For nonlinear models, maximum likelihood is a popular estimation method. 
\end{enumerate}

How to choose $f$ in step 1? Simple form (such as linear model) is easy to estimate, both imposes strong assumptions on the data. E.g.
\begin{eqnarray*}
	Y &=& \text{wage}\\
	X_1 &=& \text{experience}, \hspace{3mm} X_2 = \text{gender} (\text{1 if female, 0 if male})\\
	Y &=& \beta_0 + \beta_1X_1 + \beta_2 X_2
\end{eqnarray*}

\begin{center}
	\includegraphics[scale=0.55]{Graph1}
\end{center} 

Assumption in linear model: Same slope for males and females (experience to wage)

Examples why this might be violated:
\begin{itemize}[label={--}]
	\item Females/Males might work in different jobs with different returns to experience
	\item Discrimination against females who work part-time
	\item Relationship could be much more complex than that (e.g. wage could depend on entire employment history, not just experience) 
\end{itemize}

Bottom line: Linear model implies restrictive assumptions. (Possible fix: interaction terms)\\
Flexible forms of $f$ that impose less assumptions, but could lead to overfitting (fit training data very closely, but get problems for new (test) data). 
overfitting: i.e., accidentally modeling noise in the training data.

\section{Lecture April 25th}

$Y = f(X) + \epsilon$\\

\textbf{\tit{Parametric Estimate of $f$}}

Linear model: $f(x) = \beta_0 + \beta_1 X_1 + \ldots +\beta_p X_p$

Linear model is linear in the parameters $\beta_{ji}$ not necessarily linear in the $X_j$'s  (e.g. may have quadratic terms or interaction terms)\\
\begin{center}
	\includegraphics[scale=0.7]{Graph2}
\end{center}

\textbf{\tit{Non-parametric estimate of $f$}}\\
Try to learn the form of $f$ from the training data.\\
\id \underline{pro:} Avoids risk of misspecifying $f$\\
 \id \underline{contra:} Estimating the entire form of $f$ is ambitious, and typically requires many data points \id (i.e. large training sample)\\

\tit{Question:} Why not always choose the most flexible model that's available? (see section 2.1.3)
\underline{Inference:} Some of the very flexible models (e.g. Bagging) are hard to interpret, i.e. reveal little about the structure of the data. This leads to a trade-off between a model's flexibility and interpretability (see Figure 2.7)\\
\newpage
\underline{Subset selection lasso:} Take small set of variables from a large linear model\\
not very flexible but very interpretable\\
\rotatebox[origin=c]{180}{$\Lsh$} results are easy to interpret, because they choose the relevant variables from the data set.\\

\underline{Least squares:} Tradeoff: more flexibility $\Rightarrow$ less interpretable\\
Which models to choose depends on the data set and what models should ''show''\\

\underline{Prediction:} Choosing a very flexible model may be a bad idea due to overfitting\\

\paragraph{Supervised vs. unsupervised learning (2.1.4)}
$ $\\

\textbf{\tit{Supervised learning:}}\\
We have data on repsonse $y_i$ and parameters $x_i$ (e.g. linear regression, logistic regression, regression trees etc.)\\ 
\id Goal: Learn relationship between $Y$ and $X$\\
\id Terminology: Response $Y_i$ 'supervises' the learning process (clear what the 'correct' answer \id should be)\\

\textbf{\tit{Unsupervised learning:}}\\
We have data on $X_i$ but no response $Y_i$\\
\id Popular tool: Determine clusters of similar observations in terms of $X_i$ (Figure 2.8)\\

\tit{Example:} Identifying different groups of customers.\\
\id \rotatebox[origin=c]{180}{$\Lsh$} This class: We'll only look at supervised learning.\\

\paragraph{Regression vs. classification (2.1.5)}
$ $\\

\textbf{Quantitative} variables take on numerical values, e.g.: response time in classroom, price of a house or stock\\
\textbf{Qualitative} variables belong to one out of the $K$ classes, e.g.: categorical measure of student participation (low/medium/high), information on stock going up or down.\\

Problems with a quantitative response are called regression problems.\\
Problems with a qualitative response are called classification models.

\underline{Notes:} \begin{enumerate}
	\item Whether predictors are quantitative or qualitative is usually not important
	\item Sometimes we convert a (numerical) probability into a classifier, so distinction between regression and classification may be blurry (e.g. logistic regression)
\end{enumerate}

\paragraph{Assessing model accuracy (2.2)}
$ $\\

\tit{Question:} Many models/data sources around, which one to choose?\\

For a quantitative response, mean squared error (MSE) is the most common measure: \begin{align}
	\text{MSE} = \frac{1}{n} \sum\limits_{i = 1}^n (y_i - \hat{f}(x_i))^2, \label{Eq. 1}
\end{align}
where, $y_i =$ realization $\hat{f}(x_i) =$ its prediction\\
$\Rightarrow$ smaller MSE corresponds to better model.\\
Note that MSE in (\ref*{Eq. 1}) is computed from training data $i = 1, \ldots, n$\\
But in practice, we want to know, how our method performs for a new, previously unseen test observation $(x_0, y_0)$ that was not used for estimating our method. That is, we are interested in the \textbf{test MSE} given by \begin{align}
	E(y_0 - \hat{f}(x_0))^2 \label{Eq. 2}
\end{align}

\underline{Note:} We test $y_0$ as a random variable but $x_0$ is fixed\\
\id \rotatebox[origin=c]{180}{$\Lsh$} We know characteristics ($x_0$) of new test observations, but not the outcome ($y_0$). \\ 

\tit{How to estimate test MSE in (\ref*{Eq. 2})?} \begin{itemize}[label={--}]
	\item Use part of the data as test data
	\item What if there is not enough data to have a test sample?\\
	\id !!! Never use training MSE as an estimate of test MSE!!! 
	\item Many (all) methods can be tuned to have a small training MSE, but this does not imply a small test MSE (danger of overfitting (see Fig. 2.9)) 
	\id Green line left image: most flexible matching points too close by: sign of overfitting\\
	\id Right image: one of the two curves is training MSE (lower line). Test MSE is the red line.
\end{itemize}

\tit{Why is the training MSE lower than test MSE?}
\begin{itemize}[label={--}]
	\item model can be fitted mechanically to minimize training MSE
	\item time series data: training MSE (past), test MSE (future).\\
	\id easier to depict past than future
\end{itemize}

\newpage
\section{Lecture May 2nd}

\underline{Figure 2.9:} (In the slides)\\
(1) Left panel
\begin{itemize}[label={--}]
	\item Black curve is $f(x)$, the model from which the data was simulated
	\item Orange line, shows linear regression (rather inflexible, misses a lot of structure in data, i.e. 'underfitting')
	\item Green curve is a vary model called spline (wiggly, 'overfitting')
	\item Blue curve is a moderately flexible spline variant (just right, closest to the true curve)
\end{itemize}

(2) Right panel
\begin{itemize}[label={--}]
	\item Plot shows training and test MSE as a function of the method's flexibility
	\item Flexibility is measured as degrees of freedom (more degrees of freedom $\Leftrightarrow$ more flexible; Degrees of freedom = $\#$ parameters for linear model) 
	\item Training MSE decreases as model becomes more flexible
	\item Test MSE is U-shaped 
\end{itemize}

\tit{Why does overfitting occur?} 
\begin{itemize}[label={--}]
	\item Statistical learning methods pick up both signal and noise
	\item If a method is too flexible, it picks up a lot of noise from the training data
	\item Test data are different (different noise or $\epsilon$ values than the training data), leading to high test MSE
\end{itemize}

\tit{Ways to find 'best' model in terms of test MSE}
\begin{itemize}[label={--}]
	\item Test data (reserve part of the sample as test data)
	\item Cross validation (later in the course)
	\item Information criteria (later in the course)
\end{itemize}

\textbf{\tit{Parable of Google Flu}}\\

GFT = Google Flu Trends\\
Idea of GFT: Construct early warning indicatior of flu prevalence (Def: how often sth. occurs) based on search queries. $\rightsquigarrow$ important information, allows policy-makers to organize health policies, or shift resources from less to more affected states, etc.\\

GFT specification:\\

$\logit(\underbrace{I_t}_{\substack{\text{share of doctor visits related}\\ \text{to the flu, released by}\\ \text{CDC (official health body in the US)}}}) = \alpha \cdot \logit(\underbrace{Q_t}_{\text{share of flu-related search queries}}) + \epsilon_t$ \hspace*{4mm} ($t$ denotes time)\\

$\logit(p) = \ln(p) - \ln(1-p):$\id$(0,1) \Rightarrow \RR$ (basically for modeling convenience) \\

\tit{Question:} How to define $Q_t$? Based on (1) List/ranking of all search terms, (2) Choose amount of seach terms to include (Ginsberg et. al, Figure 1)\\
Some efforts to drop seasonal (but not flu-related) terms like college basketball, Oscar nominations.\\
\rotatebox[origin=c]{180}{$\Lsh$} Critique by Lazer et al (2014): Procedure not transparent, list of search terms not published\\

\tit{Is overfitting a big concern?} 
\begin{itemize}[label={--}]
	\item No, in the sense that the model is small (one regressor only)
	\item Yes, in the sense that there are many possible regressors (> 50 million search terms, choose list of $k$ terms, $k \in \{1, \ldots, 50.000.0000\}$) \\
	$\rightsquigarrow$ Have to choose between billions of possible specifications
\end{itemize}

\tit{What went wrong?}\\
Lazer et al, Figure 2: GFT prediction clearly and consistently overpredicts the CDC data.\\
Most likely explanation: Changes in the search engine algorithm, which led to increase in the $Q_t$ variable, which had no substantive reason.\\
$\rightsquigarrow$ Quite complex, result of both search engine technology and user behavior.\\
\newpage
\tit{Criticism by Lazer et al (2014):} 
\begin{itemize}[label={--}]
	\item GFT performance not too great to begin with
	\begin{itemize}
		\item CDC data fairly easy to predict from past, CDC data and seasonal information
		\item GFT predictions should have been compared to such a benchmark method
	\end{itemize}
	\item Changes in $Q_t$ were not incorporated/were hard to incorporate
\end{itemize}
$\rightsquigarrow$ 'Big data habris': Data can do magic just because of a large sample\\
\hspace*{4.5mm} $\rightsquigarrow$ Need to understand why a model works (substantive or statistical basis for the model's \hspace*{9.5mm} success)\\

\section{Lecture May 3rd}

\paragraph{The bias variance trade-off (2.2.2)}
$ $\\

When plotting a method's test sample MSE as a function of the method's complexity, we typically get a U-shaped pattern:(see figure 2.9 right panel)\\
Sketch: 
\begin{center}
	\includegraphics[scale=0.7]{Graph3}
\end{center}

$\rightsquigarrow$ This pattern can be explained by two competing properties of statistical learning methods: Bias and variance.
\begin{align*}
	E \{(y_0 - \hat{f}(x_0))^2\} = E\{(f(x_0) + \epsilon - \hat{f}(x_0))^2\} = E \{(f(x_0) - E(\hat{f}(x_0)) + E(\hat{f}(x_0)) + \epsilon - \hat{f}(x_0))^2\} \hspace*{17mm}\\ 
	= E(\epsilon^2) + E\{(f(x_0) - E(\hat{f}(x_0)) + E(\hat{f}(x_0)) - \hat{f}(x_0))^2\} \hspace*{82mm}\\
	 = \underbrace{E(\epsilon^2)}_{A} + \underbrace{E\{(f(x_0) - E(\hat{f}(x_0)))^2\}}_{B} + \underbrace{E\{(E(\hat{f}(x_0) - \hat{f}(x_0)))^2\}}_{C}  + \underbrace{2E\{(f(x_0) - E(\hat{f}(x_0))) (E(\hat{f}(x_0)) - \hat{f}(x_0))\}}_{D}
\end{align*}
$\epsilon$ is part of the test sample $\Leftrightarrow$ independent of $\hat{f}(x_0)$ (part of the training sample)\\

$D = 2E\{(f(x_0) - E(\hat{f}(x_0))) (E(\hat{f}(x_0)) - \hat{f}(x_0))\} = 2E\{(f(x_0) - \hat{f}(x_0)) \underbrace{(\hat{f}(x_0) - \hat{f}(x_0))}_{=0}\} = 0$\\
$A = Var(\epsilon) \geq 0$\\
$B = E\{(\underbrace{f(x_0)}_{\text{optimal prediction}} - \underbrace{E(\hat{f}(x_0))}_{\text{expected value of prediction}})^2\} =$ Bias of $\hat{f}(x_0) \geq 0$\\

$C = E\{(E(\hat{f}(x_0) - \hat{f}(x_0))^2\} = Var \{\hat{f}(x_0)\} \geq 0$\\

Three components of expected test MSE\\
(1) \textbf{Var$(\epsilon)$} $\rightsquigarrow$ irreducible error, cannot do anything about this one $\rightsquigarrow$ Measure of difficulty of prediction problem\\
(2) \textbf{Bias of $\hat{f}(x_0):$} Expresses the error that we make by approximating a (possibly) complex real-life problem with a statistical model. E.g. when the true curve $f$ is nonlinear, then linear model $\hat{f}$ will always lead to a biased estimate (even if we had a huge training data set).\\
$\rightsquigarrow$ In general, more flexible models will have smaller bias.\\
Intuition: Flexible methods can produce various curves $f$ (e.g. linear or nonlinear, smooth or rough), depending on the data. 

(3) \textbf{Variance of $\hat{f}(x_0)$:} Expresses how much $\hat{f}$ would change if we used a different training data set to estimate $\hat{f}$. If $Var \{\hat{f}(x_0)\}$ is high, this means that the SL method is fragile (small changes in training data can lead to large changes in $\hat{f}$).\\
$\rightsquigarrow$ in general, flexible methods tend to have higher variance. \\
\begin{center}
	\includegraphics[scale=0.7]{Graph4}
\end{center}

\begin{itemize}[label={--}]
	\item When using a more flexible method, we increase the variance but reduce the bias
	\item Challenge is to choose the optimal compromise that minimizes test MSE
\end{itemize}
$\rightsquigarrow$ This challenge is called the bias-variance trade-off. 
\newpage
\paragraph{The classification setting (2.2.3)}
$ $\\

$\rightsquigarrow$ concepts like Bias to Variance Tradeoff transfer classification setting, but we need some modification since $y_i$ is now qualitative.\\

Suppose we seek to estimate $f$ based on a training observations $\{(x_1, y_1), \ldots, (x_n, y_n)\}$, where the $y_i$'s are now qualitative.\\
\underline{Training error rate} is defined as 
\begin{align*}
	\frac{1}{n} \sum\limits_{i =1}^n \mathbbm{1}(y_i \neq \hat{y}_i)
\end{align*}
where $\hat{y_i}$ is the predicted class label for observation $i$, based on $\hat{f}(x_i)$. Furthermore, 
\begin{align*}
	\mathbbm{1}(y_i \neq \hat{y}_i) = \begin{cases}
		1 \text{ if } y_i \neq \hat{y_i} \text{  (i.e. wrong classification)}\\
		0 \text{ if } y_i = \hat{y_i} \text{ (i.e., correct classification)}
		\end{cases}
\end{align*}
Training error rate gives traction of wrong classifications. As in the regression setting, we are most interested in the test sample performance of a SL method. The latter is captured by the test error rate, which for a new observation $(x_0, y_0)$, is given by
\begin{align*}
	P(y_0 = \hat{y_0}), 
\end{align*}
i.e the probability that the SL method gives a wrong classification of the new observation.\\ $\rightsquigarrow$ should be as small as possible. 
\newpage
\section{Lecture May 9th} 

Measures of (test sample) error:\\
\hspace*{5mm} Mean (expected) squared error \begin{align*}
	E (y_0 - \hat{f} (x_0))^2 \hspace*{10 mm} \text{numerical/continuous response} 
\end{align*}
\hspace*{5mm} $(\underbrace{x_0}_{\text{known}}, \underbrace{y_0}_{\text{unknown}})$ is a new data point that's not used for model fitting.\\

\hspace*{5mm} Probability of false classification: \begin{align}
	\hspace*{4mm} P(y_0 \neq \hat{y}_0), \hspace*{8mm} \text{qualitative response (classification)} \label{Eq. 3}
\end{align}

\hspace*{5mm} where $y_0 \in \{0,1\}$ is the observed class label, and $\hat{y}_0 \in \{0,1\}$ is the predicted label.\\
\hspace*{90mm} \rotatebox[origin=c]{180}{$\Lsh$} Transfers to the case of > 2 classes



\textbf{\tit{Bayes classifier}}\\

Can be shown: Test error rate in (\ref*{Eq. 3}) is minimized by assigning each observation to the most likely class, given its predictor values. I.e., assign test observation with predictor vector $x_0$ to the class $j$ for which \begin{align*}
	P(Y = j| X = x_0)
\end{align*}
is largest. $\rightsquigarrow$ ''Bayes classifier'' (BC)\\
If there are only two classes: Assign to class '1' if \begin{align*}
 P(Y = 1| X = x_0) > 0.5
\end{align*}

BC produces the lowest possible error rate ('Bayes error rate'), given by \begin{align*}
	1 - E\left(\max\limits_{j} P(Y = j|X)\right),
\end{align*}
this corresponds to the irreducible error in regression.\\
\newpage
\textbf{\tit{$K$-Nearest Neighbors}}\\

Problem with the BC: Depends on the unknown true probability. $\rightsquigarrow$ Need to estimate it from training data. One approach for doing this:\\
\textbf{$K$-nearest neighbors (KNN)} classifier.\\

\underline{Idea:} Find $K$ points in the training data that are closest to $x_0$.\\
Denote this set as $\mathcal{N}_0$. Then let \begin{align*}
	\hat{P}(Y = j| X = x_0) = \frac{1}{k} \sum\limits_{i \in \mathcal{N}_0} \mathbbm{1}(y_i = j),
\end{align*}
i.e. the fraction of neighbors that belong to class $j$.

\tit{Example:} Figure 2.14
\begin{itemize}[label={--}]
	\item Training data given by six blue and six orange observations
	\item $x_0$ marked by black cross
	\item  Example: $k = 3$ $\rightsquigarrow$ Three NNs shown (green circle)\begin{align*}
		\begin{rcases*}
		\hat{P}(Blue| X = x_0) = 2/3\\
		\hat{P}(Orange | X = x_0) = 1/3
		\end{rcases*} \text{Classify as blue (no weighting of neighbors)}
	\end{align*}
	Right panel shows KNN decision boundary
\end{itemize}

\tit{What's the role of $K$?} \begin{itemize}[label={--}]
	\item Low value of $K$: Very flexible method $\rightsquigarrow$ Low bias, high variance\\
	\hspace*{22mm} \rotatebox[origin=c]{180}{$\Lsh$} Decision boundary becomes wiggly (small change in $x_0$ can change class)
	\item High value of $K$: Inflexible method $\rightsquigarrow$ High bias, low variance.\\
	\hspace*{25mm}  \rotatebox[origin=c]{180}{$\Lsh$} Decision boundary becomes very smooth
\end{itemize}

Extreme example: $K = n$ (= \# traning observations) \begin{itemize}[label={--}]
	\item Always predict majority class ($\mathcal{N}_0$ is the complete sample)
\end{itemize}

Low $K$ means low training error (= zero if $K = 1$)\\
More importantly: Test error! 
\begin{center}
	\includegraphics[scale=0.7]{Graph5}
\end{center}
\underline{$K$-nearest neighbor example from the book Chapter 2 Problem 7:}\\

(a) Compute euclidean distance:\\ $\begin{matrix}
	Obs & eucl. distance & Color\\
	1 & 3 & R\\
	2 & 2 & R\\
	3 & \sqrt{10} & R\\
	4 & \sqrt{5} & G\\
	5 & \sqrt{2} & G\\
	6 & \sqrt{3} & R\\
\end{matrix}$\\

(b) $\hat{P}(Green| X = x_0) = 1$ for $K = 1$\\
	use only observation 5, i.e. $\mathcal{N}_0 = \{5\}$\\
	
(c) $K = 3:$\\
	$\hat{P}(Green| X = x_0) ) = 1/3$\\
	use obserservation $2,5$ and $6$, i.e. $\mathcal{N}_0 = \{2,5,6\}$\\

(d) If decision boundary is complex, need a flexible model\\
	\rotatebox[origin=c]{180}{$\Lsh$} Small value for $K$.\\
 
\section{Lecture, May 17th}

\tit{\textbf{Linear regression}}\\

\underline{Linear regression} is the standard approach for supervised learning with a quantitative response.\\
\id $\rightsquigarrow$ useful starting point before discussing more complicated SL methods.\\

Simplest case: One predictor variable \begin{align*}
	Y = \beta_0 + \beta_1 X + \epsilon
\end{align*} \begin{itemize}
	\item $\beta_0, \beta_1$ are unknown parameters $\rightsquigarrow$ need to estimate them
	\item Prediction: $\hat{Y} = \hat{\beta}_0 + \hat{\beta}_1 X$
\end{itemize}
\tit{Example:}\id $X$ = education (years of schooling), $Y$ = Income (Euros)\\
$\rightsquigarrow$ Prediction $\hat{Y}$ is a linear function of $X$\\

\begin{align*}
	\includegraphics[scale=0.55]{Graph6}
\end{align*}

\newpage
\paragraph{Estimating the coeffictions (3.1.1)}
$ $\\

Training data $(X_1, Y_1), \ldots, (X_n, Y_n)$.\\
Goal: Find estimates $\hat{\beta}_0$, $\hat{\beta}_1$ that fit the data, i.e. such that $e_i = (y_i - \hat{\beta}_0 - \hat{\beta}_1 X_i)$ is 'small' for $i = 1, \ldots, n$.\\
 $\rightsquigarrow e_i$ is the residual (difference between observation and prediction) for case $i$.\\
Residual sum of squares (RSS) is given by \begin{align*}
	\text{RSS} = e_1^2 + e_2^2 + \ldots + e_n^2
\end{align*}
Which estimates $(\hat{\beta}_0, \hat{\beta}_1)$ minimize RSS?\\

Optimization problem: \begin{align*}
	\min\limits_{b_0, b_1} \text{ RSS} = \sum\limits_{i = 1}^n (Y_i - b_0 - b_1 X_i)^2 \equiv \text{RSS } (b_0, b_1)
\end{align*}
Note, that $b_0$ and $b_1$ are estimators for $\beta_0$ and $\beta_1$ respectively. Solving the problem: 
\begin{eqnarray*}
	\frac{\dif \text{RSS}(b_0, b_1)}{\dif b_0} &=& -2 \sum\limits_{i = 1}^n (Y_i - b_0 - b_1 X_i) \overset{!}{=} 0\\
	\Leftrightarrow \id \id \underbrace{\sum\limits_{i = 1}^n Y_i}_{\equiv n \overline{Y}} &\overset{!}{=}& n b_0 + b_1 \underbrace{\sum\limits_{i = 1}^n X_i}_{= n \overline{X}}\\
	\Leftrightarrow \id \id \id  \hat{\beta}_0 &=& \overline{Y} - \hat{\beta}_1 \overline{X}
\end{eqnarray*}

\begin{eqnarray*}
	\frac{\dif RSS(b_0, b_1)}{\dif b_1} &=& -2 \sum\limits_{i = 1}^n X_i (Y_i - b_0 - b_1 X_i)\overset{!}{=} 0\\
	\Leftrightarrow \id \sum\limits_{i = 1}^n X_i Y_i &=& b_0  \sum\limits_{i = 1}^n X_i + b_1 \sum\limits_{i = 1}^n (X_i)^2\\
\end{eqnarray*}
{plug in $\hat{\beta}_0$
\begin{eqnarray*}
	\sum\limits_{i = 1}^n X_i Y_i &=& (\overline{Y} - \hat{\beta}_1 \overline{X}) n \overline{X} +  \hat{\beta}_1 \sum\limits_{i = 1}^n (X_i)^2 = n \overline{X} \overline{Y} + \hat{\beta}_1 (\sum\limits_{i = 1}^n (X_i)^2 - n \overline{X}^2)\\
	\Leftrightarrow \id \underbrace{\frac{1}{n} \sum\limits_{i = 1}^n X_iY_i - \overline{X} \overline{Y}}_{Cov(X_i, Y_i)} &=& \hat{\beta}_1 \underbrace{\frac{1}{n} \sum\limits_{i = 1}^n X_i^2 - \overline{X}^2)}_{Var(X_i)}\\
	\Leftrightarrow \id \hat{\beta}_1 &=& \frac{\frac{1}{n} \sum\limits_{i = 1}^n X_i Y_i - \overline{X} \overline{Y}}{\frac{1}{n} \sum\limits_{i = 1}^n (X_i^2 - \overline{X}^2)} = \frac{Cov(X_i, Y_i)}{Var(X_i)}
\end{eqnarray*}

\begin{itemize}
	\item Estimate $\hat{\beta}_1$ is positive whenever there's a positive correlation between $X_i$ and $Y_i$ in the sample.
	\item Average residual given by \begin{align*}
		\frac{1}{n} \sum\limits_{i = 1}^n e_i = \frac{1}{n} \sum\limits_{i = 1}^n (Y_i - (\underbrace{\overline{Y} - \hat{\beta}_1 \overline{X}_i}_{= \hat{\beta}_0}) - \hat{\beta}_1 X_i) = \frac{1}{n} \sum\limits_{i = 1} (Y_i - \overline{Y}) = 0
	\end{align*}
	$\rightsquigarrow$ by construction, average residual is zero
\end{itemize}
$\hat{\beta}_0$, $\hat{\beta}_1$ shown above are called ordinary least squares (OLS) estimates\\
$\rightsquigarrow$ can be shown: $E(\hat{\beta_0}) = \beta_0$, $E(\hat{\beta}_1) = \beta_1$ (i.e. estimates are unbiased)\\
$\rightsquigarrow$ if you simulate many data sets of size $n$, then both $\hat{\beta}_0$ and $\hat{\beta}_1$ will be correct on average (great programming exercise)\\

Which factors affect accuracy of OLS estimator?\\
\id \id \rotatebox[origin=c]{180}{$\Lsh$} Ideally, want $Var(\hat{\beta}_0)$ and $Var(\hat{\beta}_1)$ to be small.\\
This is the case if, \begin{itemize}
	\item The sample size, $n$, is large
	\item The error variance (of $\epsilon$), $\sigma^2$, is small 
	\item The variance of $X_i$, $Var(X_i)$, is large
\end{itemize}
$\rightsquigarrow$ \underline{Note:} RSS is minimized by construction, hence not a good measure of test sample performance.\\

\paragraph{Multiple Linear Regression (3.2)}
$ $\\

Often want to consider more than one predictor variable \begin{align*}
	Y = \beta_0 + \beta_1 X_1 + \beta_2 X_2 + \ldots + \beta_p X_p + \epsilon \id \id  \text{(p predictor variables)}
\end{align*}
e.g. \begin{eqnarray}
	Y &=& \text{income}\\
	X_1 &=& \text{years of schooling}\\
	X_2 &=& \text{gender (e.g. 1 if female, 0 if male)}\\
	X_3 &=& \text{job experience (in years)}\\
	X_4 &=& \text{location1 (e.g. 1 if east, 0 if west)}\\
	X_5 &=& \text{location2 (e.g. 1 if urban area, 0 if not)}\\
	X_6 &=& \text{online courses (e.g. 1 if taken MOOC)}\\
	&\vdots&
\end{eqnarray}
\tit{How to interpret parameters?}\\
\id $\rightsquigarrow$ E.g., $\beta_3$? 
\id $\rightsquigarrow$ Change in predicted income if job experience increases by one year but all other variables stay constant\\

OLS estimation: \begin{align*}
	\text{minimize}_{b_0, \ldots, b_p} \text{ RSS} = \sum\limits_{i = 1}^n (Y_i - b_0 - b_1X_{i1} - \ldots - b_p X_{ip})^2
\end{align*}
\id $\rightsquigarrow$ Problem can be solved using matrix Algebra\\
\id $\rightsquigarrow$ Implemented in statistical software, e.g. R function 'lm'\\

\tit{Note:} Variable may be important in simple linear regression (1 predictor) but may drop out/be irrelevant in multiple regression.\\
E.g.: In regression of $Y$ on $X_6$ only, might find strong positive relation (but might vanish in multiple regression)\\
\newpage
\tit{Model fit} (i.e. how well a model matches  the data)\\
Most popular measure: $R^2$, given by \begin{align*}
	R^2 = 1 - \frac{\text{RSS}}{\text{TSS}}, \id \text{TSS} = \sum\limits_{i = 1}^n (Y_i - \overline{Y})^2
\end{align*}

\tit{Properties:} \begin{itemize}
	\item $0 \leq R^2 \leq 1$, \begin{itemize}
		\item $RSS = 0 \Rightarrow$ training sample predictions are perfect $\Rightarrow R^2 = 1$
		\item \text{RSS = TSS} $\Rightarrow \hat{Y}_i = \overline{Y}$, $i = 1, \ldots, n$ ($X$ variables not at all helpful for predicting $Y$) $\rightsquigarrow R^2 = 0$
	\end{itemize} 
	\item If you include more variables into the model, then $R^2$ increases by construction
	\begin{itemize}
		\item TSS stays the same
		\item RSS goes down by construction (more degrees of freedom) 
	\end{itemize}
	\item Will look at better measures of model fit (e.g. adjusted $R^2$, Akaike information criterion) $\rightsquigarrow$ account for the number of predictors $p$, that are used in the model
\end{itemize}

\section{Lecture May 23rd} 

Linear regression model \begin{align*}
	Y = \beta_0 + \beta_1X_1 + \ldots + \beta_p X_p + \epsilon
\end{align*}
$R^2 = 1 - \frac{\text{RSS}}{\text{TSS}}$\\
Note: OLS estimator is designed to minimize RSS $\Leftrightarrow$ maximize $R^2$. It can be shown that $R^2 = (\underbrace{Cor(Y_i, \hat{Y}_i)}_{\text{Correlation in training sample}})^2$\\

Main problem with $R^2$: Increases mechanically when using more predictors.\\
\id $\rightsquigarrow$ We'll look at better ways to measure model fit\\

\tit{Prediction in linear model}\\
...given by $\hat{Y} = \hat{\beta}_0 + \hat{\beta}_1 X_1 + \ldots + \hat{\beta}_p X_p$ $\neq$ true expected value (if model is not linear) \begin{align*}
	f(X) = \beta_0 + \beta_1X_1 + \ldots + \beta_p X_p 
\end{align*}
Ways to quantify prediction uncertainty \begin{enumerate}
	\item \underline{Confidence interval}: Describe how close $\hat{Y}$ is to $f(X)$. E.g. $95 \%$ CI should contain $f(X)$ with probability $95  \%$\\
	\id $\rightsquigarrow$ Estimation uncertainty about $\beta_0, \beta_1, \ldots$
	\item \underline{Prediction interval}: Describes uncertainty about both $f(X)$ and $\epsilon$. E.g., $95 \%$ PI should contain $Y$ with probability $95 \%$
\end{enumerate}

\tit{Note}: \begin{itemize}
	\item PI generally wider (longer) than CI
	\item As $n \rightarrow \infty$, CI shrinks to true value $f(X)$ (i.e. CI has lenght zero, no estimation uncertainty left). 
	\item As $n \rightarrow \infty$, PI only covers uncertainty about $\epsilon$ (does not collapse to a length zero), there can still be a lot of uncertainty left) 
\end{itemize}
\newpage
\paragraph{Other considerations in the linear regression model (3.3)}

\paragraph{Qualitative predictors (3.3.1)}
$ $\\

E.g. prediction model for the amount somebody donates to an online charity project.\\
Possible qualitative predictors in that example: \begin{itemize}
	\item gender
	\item employment status
	\item income category (often used as categorical variable for data privacy reasons)
	\item profession (say, $K$ different groups) 
	\item political orientation
\end{itemize}
$\rightsquigarrow$ Question: How to include these variables into a model?\\

\underline{Case $A$: Only two levels}\\

Example: Unemployed Yes/No \\
Create a new variable \begin{align*}
	X_i = \begin{cases}
	1 \text{ if person $i$ is unemployed}\\
	0 \text{ if not} 
	\end{cases}
\end{align*}
and use this variable in a regression model.\\
E.g. \begin{align*}
	Y_i = \beta_0 + \beta_1 X_i + \epsilon_i = \begin{cases}
	\beta_0 + \beta_1 \text{ if person $i$ is unemployed}\\
	\beta_0 \text{ if not}
	\end{cases}
\end{align*}

\underline{Charity example}:\\
\id $\beta_0 =$ Mean contribution of employed persons\\
\id $\beta_1 =$ Difference in (mean) contribution for unemployed vs. employed persons\\
\newpage
\tit{Note}: \begin{itemize}
	\item Coding of $X_i$ is arbitrary: Could also set $X_i$ to 1 for employed persons\\
	\id \rotatebox[origin=c]{180}{$\Lsh$} Interpretation of parameters would change\\
	\id \rotatebox[origin=c]{180}{$\Lsh$} RSS statistic/$R^2$ would stay the same
\end{itemize}
\underline{Case $B$: More than two levels}\\
E.g., income category (low/middle/high) $\rightsquigarrow$ 3 levels $\rightsquigarrow$ Need to create two dummy variables \begin{align*}
	X_{i1} = \begin{cases}
	 1 \text{ if $i$ is in the middle income category}\\
	 0 \text{ if not}
	\end{cases}\\
	X_{i2} = \begin{cases}
	1 \text{ if $i$ is in the high-income category}\\
	0 \text{ if not}
	\end{cases}\\
\end{align*}
Then\\
$Y_i = \beta_0 + \beta_1 X_{i1} + \beta_2 X_{i2} + \epsilon_i = \begin{cases}
	\beta_0 + \epsilon_i \id \id \text{low-income}\\
	\beta_0 +  \beta_1 + \epsilon_i \id \id \text{middle-income}\\
	\beta_0 + \beta_2 + \epsilon_i \id \id \text{high-income}\\
\end{cases}$ \begin{itemize}
	\item $\beta_0$ = mean contribution for low-income
	\item $\beta_1$ = difference in mean contribution for middle vs. low income
	\item $\beta_2$ = difference in mean contribution for high vs. low income
\end{itemize}
\id low-income category is the baseline category\\

Example: Linear models in R \begin{itemize}
	\item lm($y \sim x$, data = d) runs linear regression of $y$ on $x$, whereby both are in data set d.
	\item If $x$ is a factor variable, then R automatically codes the ($K-1$) binary variables we discussed, where $K = \#$ of levels for $x$.
\end{itemize}

\section{Lecture May 24th}

Yesterday's example: \begin{align*}
	\underbrace{Y}_{\stackrel{\text{fundraising}}{\text{contribution}}} = \beta_0 + \beta_1X_1 + \ldots + \beta_pX_p + \epsilon
\end{align*}
where the $X_i$ are personal charactersitics (age, education, political orientation)

\begin{align*}
	\includegraphics[scale=0.55]{Graph7}
\end{align*}

Possible violation of A2: Slope of regression line differs across subgroups

\begin{align*}
	\includegraphics[scale=0.55]{Graph8}
\end{align*}

Model 2 yields two different regression lines: \begin{itemize}
	\item $Y = \beta_0 + \beta_1 X_1 + \epsilon$ \id \id \id \id \id without university education
	\item $Y = (\beta_0 + \beta_2) + (\beta_1 + \beta_3)X_1 + \epsilon$ \id with university education
\end{itemize}
$\rightsquigarrow$ Note: Model 2 is more complex than model 1, hence lower bias but higher variance.\\
Possible violation of A1: Relation and $Y$ and $X_1$ is not linear. 

\begin{align*}
	\includegraphics[scale=0.60]{Graph9}
\end{align*}
$\rightsquigarrow$ Can we cover nonlinear relations in the linear model?\\
Possible fixes: Define age categories, i.e. qualitative variable \begin{align*}
	X_2 = \begin{cases}
	1 \text{ if } X_1 < 18\\
	2 \text{ if } 18 \leq X_1 < 24\\
	3 \text{ if } 24 \leq X_1 < 30\\
	\vdots
	\end{cases}
\end{align*}

\begin{align*}
	\includegraphics[scale=0.55]{Graph10}
\end{align*}

\tit{Pro}: More flexible\\
\tit{Con}: \begin{enumerate}
	\item Implicitly adds many variables (K-1 binary variables for K age groups) $\rightsquigarrow$ complex model
	\item Piecewise constant (no variation in subgroup) $\rightsquigarrow$ can be generalized to the piecewise linear model. 
	\item Non-differentiable
	\item Uses no information from 'neighboring' age groups (e.g., coefficient for group 54-60 uses no information from people aged 48-54)
\end{enumerate}
\newpage
\tit{Better alternatives}: 

\begin{enumerate}
	\item Use linear model with polynomials of Age
	\begin{align*}
		\includegraphics[scale=0.55]{Graph11}
	\end{align*}
	\item Use a non-linear/non-parametric model (e.g. K-nearest neighbors) 
\end{enumerate}

\paragraph{Classification (Chapter 4)}
$ $\\
\begin{itemize}
	\item Predicting qualitative (aka categorical) variable is often called 'classification' $\rightsquigarrow$ Need to assign each observation to one category, or class. 
	\item  \underline{Examples:} \begin{itemize}
		\item Predict education of a person (e.g. vocational degree, BSc, MSc, ...) 
		\item Income categories (if exact numerical income is not available) 
		\item Acceptance to a university program
	\end{itemize}
	\item Classification methods often model the probability of each category, and then classify based on the probabilities
	\item Methods we discuss here: \begin{itemize}
		\item Logistic regression
		\item Linear Discriminant Analysis
		\item K-nearest Neighbors
	\end{itemize}
	\item Later in the course (more computer-intensive): \begin{itemize}
		\item Trees
		\item Random Forests
		\item Boosting 
	\end{itemize}
\end{itemize}

\paragraph{Why not linear regression (4.2)}
$ $\\

\tit{Example}: Predict a person's choice of bus vs. train vs. car for commuting to work/university.\\
\id Suppose we have a qualitative variable $Y$ for transportation choice, i.e. \begin{align*}
	Y = \begin{cases}
	1 \text{ if train}\\
	2 \text{ if car}\\
	3 \text{ if bus}
	\end{cases}
\end{align*}
\id What if we used least squares/linear model to predict $Y$? \begin{align*}
	Y = \beta_0 + \beta_1\underbrace{X_1}_{\text{age}} + \ldots + \beta_p\underbrace{X_p}_{\stackrel{\text{commuting}}{\text{distance}}}  + \epsilon
\end{align*}

\id Here, the linear model would be based on two unrealistic assumptions \begin{itemize}
	\item Ordering in transportation choices (train < car < bus???) 
	\item Numerical interpretation  (train = 1/2, car = 1/3 bus???)
\end{itemize}
\id $\rightsquigarrow$ Linear model is not a reasonable choice here\\

The situation is somewhat better (for the linear model) when there are only two levels e.g. \begin{align*}
	Y = \begin{cases}
		1 \id \text{ if person chooses car}\\
		0 \id \text{ if not} 
	\end{cases}
\end{align*}

Here the ordering assumption is less implausible (can be viewed as likelihood of choosing 'car') $\rightsquigarrow$ Linear model would predict 'car' if $\hat{Y} > 0.5$.\\
One can also show: Predictions stay the same if we swap the labels (i.e. 0 if car, 1 if no car).\\ 

\tit{Con}: Can have predictions $\hat{Y} < 0$ or $\hat{Y} > 1$, hence interpretation as probability no longer holds.\\
\id \id \rotatebox[origin=c]{180}{$\Lsh$} Better alternative: Logistic regression\\
Consider a binary response $Y \in \{0,1\}$, e.g. $Y = \begin{cases}
1 \text{ if email is spam}\\
0 \text{ if not}
\end{cases}$\\
For simplicity, let's consider a single predictor $X =$ Percentage of capital letters in subject line\\
Logistic regression model sets \begin{align*}
	P(Y = 1| X) = P(X) = \frac{\exp(\beta_0 + \beta_1 X)}{1 + \exp(\beta_0 + \beta_1 X)} \in (0,1) 
\end{align*}

\begin{align*}
	\includegraphics[scale=0.55]{Graph12}
\end{align*}

Note that \begin{align*}
	\underbrace{\log\left(\frac{p(x)}{1 - p(x)}\right)}_{\text{log-odds ration } \in (- \infty, \infty)} = \beta_0 + \beta_1 X
\end{align*}

\section{Lecture May 30th} 

\paragraph{Logistic Regression Mode (4.3.1)}
$ $\\

Binary response $Y \in \{0, 1\}$, e.g.\\
 $Y = 1$ referendum passes (quorum + majority in favor),\\
 $Y = 0$ referendum fails.

Regressor $X$, e.g. \begin{itemize}
	\item share of male voters
	\item average education level of voters
	\item voter turnout (percentage of population that participates in referendum) 
\end{itemize}

\tit{Aim}:\id Model $Pr(Y= 1|X) = p(x)$\\
\id Logistic regression model $p(x) = \frac{\exp(\beta_0 + \beta_1 X)}{1 + \exp(\beta_0 + \beta_1 X)} \in (0,1)$\\
\id $\log \frac{p(x)}{1 - p(x)} = \beta_0 + \beta_1 X$ $\in (-\infty, \infty)$\\

Furthermore, \begin{eqnarray*}
	\frac{\dif p(x)}{\dif x} &=& \frac{\beta_1 \exp(\ldots) \{1 + \exp(\ldots)\} - \beta_1 \exp(\ldots) \exp(\ldots)}{(1 + \exp(\beta_0 + \beta_1X))^2}\\
	&=& \frac{\beta_1 \exp(\ldots)}{(1+ \exp(\ldots))^2}  \begin{cases}
	 > 0 \text{ if } \beta_1 > 0\\
	 = 0 \text{ if } \beta_1 = 0\\
	 < 0 \text{ if } \beta_1 < 0
\end{cases}
\end{eqnarray*}

E.g. if we find, that $\beta_1 > 0$, that implies positive association between $X$ and $p(X)$\\
$\rightsquigarrow$ can interpret sign of $\beta_1$!\\
\tit{Note:}\\
$\frac{\dif p(x)}{\dif x}$ (size of effect of $X$) depends on $X$ $\rightsquigarrow$ Different for each observation in sample. $\rightsquigarrow$ can not easily interpret size of $\beta_1$!\\
\newpage
\paragraph{Estimating the regression coefficients (4.3.2)}
$ $\\

\begin{itemize}
	\item For OLS/linear model: There exists an explicit formula for the coefficient estimates. $\rightsquigarrow$ convenient situation, arises because the model is linear in $\beta$
	\item Logistic model no longer linear in $\beta$ $\rightsquigarrow$ cannot use the OLS estimator
	\item  Instead, use maximum likelihood (ML) estimator \begin{itemize}
		\item Idea: Look for parameter values $\beta_0, \beta_1$ such that the predicted probability corresponds as closely as possible to the oberserved outcome (i.e., probability should be as small as possible if $Y = 0$ and as large as possible if $Y =1$)
	\item Formally: Use likelihood function \begin{align*}
		\textit{l}(\beta_0, \beta_1) = \underbrace{\prod\limits_{i: Y_i = 1} p(X_i)}_{\text{observations for which $Y = 1 \rightsquigarrow$ want $p$ as large as possible}}  \id \underbrace{\prod\limits_{j: Y_i = 0} (1 - p(X_j))}_{\text{observations for which $Y = 0\rightsquigarrow$ want $p$ as small as possible}}
	\end{align*}
	where $p(X_i)$ and $p(X_j)$ depend on $\beta_0, \beta_1$
	\end{itemize}
\end{itemize}

\tit{Note}: \begin{itemize}
	\item In practice, we use the log likelihood function, which has better numerical properties. 
	\item Parameters $\hat{\beta}_0, \hat{\beta}_1$, that maximize $\ln \textit{l}(\beta_0, \beta_1)$ are called \underline{maximum likelihood} estimator \begin{itemize}
		\item Good theoretical properties (consistent, low variance)
		\item Often easy to use in practice (e.g., $R$'s glm function)
	\end{itemize}
\end{itemize}

\paragraph{Making predictions (4.3.3)}
$ $\\

Example in book: $Y \in \{0,1\}$, $Y = 1$ means that somebody defaults on a credit.\\
\id \id $\rightsquigarrow$ Want to predict the probability of default, given $X$ ('balance')\\
In practice: Run MLS to get $\hat{\beta_0}, \hat{\beta_1}$, then we use \begin{align*}
	\hat{p}(X) = \frac{\exp(\hat{\beta_0} + \hat{\beta_1} X)}{1 + \exp(\hat{\beta_0}, \hat{\beta_1}X)})
\end{align*}

\paragraph{Multiple logistic regression (4.3.4)}
$ $\\

Consider using several predictors $X_1, \ldots, X_p$, such that \begin{eqnarray*}
	\log \frac{p(x)}{1 - p(X)} &=& \beta_0 + \beta_1 X_1 + \ldots + \beta_p X_p\\
	\Leftrightarrow \id p(X) &=& \frac{\exp(\beta_0 + \beta_1 X_1 + \ldots + \beta_p X_p)}{1 + \exp(\beta_0 + \beta_1 X_1 + \ldots + \beta_p X_p)}\\
\end{eqnarray*}

Credit example: $X_1$ = balance, $X_2$ = income, $X_3$ = student (yes/no)\\
\id $\rightsquigarrow$ Especially helpful if predictors $X_1, \ldots, X_p$ are correlated\\
Credit example (Figure 4.3): \begin{itemize}
	\item Students have higher balance values on average
	\item For \underline{given} balance, students are less likely to default. 
\end{itemize}
$\rightsquigarrow$ multiple regression takes care of these effects by allowing for 'all else equal' interpretation.\\
Yes/no variable as regressor (e.g. $X_3$ = student yes/no) \begin{align*}
	\hat{p}(X) =  \begin{cases}
	\frac{\exp(\hat{\beta}_0 + \hat{\beta}_1X_1 + \hat{\beta}_2X_2)}{ 1+ \exp(\hat{\beta}_0 + \hat{\beta}_1X_1 + \hat{\beta}_2X_2)} \id \id \text{non-students}\\
	\frac{\exp(\hat{\beta}_0 + \hat{\beta}_1X_1 + \hat{\beta}_2X_2 + \hat{\beta}_3)}{ 1+ \exp(\hat{\beta}_0 + \hat{\beta}_1X_1 + \hat{\beta}_2X_2 + \hat{\beta}_3)}  \id \id \text{students}
	\end{cases}
\end{align*}

\paragraph{Logistic regression with > 2 response classes (4.3.5)}
$ $\\

E.g. three categories: Default vs. renegotiate vs. pay\\
\id $\rightsquigarrow$ Model each of \underline{three} probabilities via logistic regression\\
\id $\rightsquigarrow$ Can be done, but not covered in this course\\
\id $\rightsquigarrow$ Instead, look at discriminant analysis (classification method that generalizes more easily\\
\id \id to > 2 classes)\\

\section{Lecture June 7th}

Logistic regression: Model $P(Y = 1|X)$

Alternative method: Linear discriminant analysis (LDA) \begin{itemize}
	\item Model $f(X|Y = 1)$
	\item Model $f(X|Y = 0)$
\end{itemize}
$\rightsquigarrow$ then flip around to get $P(Y = 1| X)$

Why do we need that? \begin{itemize}
	\item Two situations where logistic regression doesn't work well:\\
	 1) If classes ($Y = 0$ and $Y = 1$) are well separated, LR coefficients may be unstable,\\
	 2) if the number of observations in training sample is small
	\item LDA can be easily generalized to $K$ classes ($Y = 1, 2, \ldots, K$)
\end{itemize} 
	
\tit{\textbf{Using Bayes' Theorem for Classification}}\\

Setup: $Y$ is a qualitative variable with $K \geq 2$ unordered classes, e.g. \begin{itemize}
	\item $Y = 1, 2, \ldots, K$ different types of images
	\item $Y = 1, 2, \ldots, K$ different types of cars (marketing) 
\end{itemize}
Let $\pi_k$ be the prior probability (before seeing $X$) that $Y = \underbrace{K}_{\text{one of the classes}}$\\
Let $f_K(X)$ denote the \underline{density function} of $X$ for an observation from the $K$-th class\\
$\rightsquigarrow f_K(x)$ large if there's a high probability that an observation from class $K$ has $X \approx x$\\ 
E.g. $X$ = income, $Y \in \{\underbrace{\text{cheap car}}_1, \underbrace{\text{medium-price car}}_2, \underbrace{\text{expensive car}}_3\}$
\begin{align*}
	\includegraphics[scale=0.55]{Graph13}
\end{align*}
	
\underline{Bayes' theorem} states that \begin{align}
	Pr(Y = K| X = x) = \frac{\pi_K f_k(x)}{\sum\limits_{l = 1}^K \pi_l f_l(x)}  \label{Eq. 4}
\end{align}	
$\rightsquigarrow$ 'Posterior Probability' after seeing $X$\\
$\rightsquigarrow$ Can estimate $P_K(x) \equiv Pr(Y = K|X = x)$ by inserting estimates for $\{\pi_l\}_{l = 1}^K$ and $\{f_l\}_{l = 1}^K$ into (\ref*{Eq. 4})
\begin{itemize}
	\item Estimating the $\{\pi_l\}$ is easy: Can use fraction of training data that fall into each category
	\item Estimating the densities $\{f_l(x)\}$ is more challenging, but can be simplified with adding assumptions
\end{itemize}
$\Rightarrow$ Next: Discuss approaches for estimating the terms in (\ref*{Eq. 4}), with the aim to approximate the Bayes classifier.\\
\newpage
\paragraph{Linear discriminant analysis for $p = 1$ (4.4.2)}
$ $\\

Suppose that $f_K(x)$ is normal with \begin{align*}
	f_K(x) = \frac{1}{\sqrt{2 \pi \sigma^2}} \exp\{- \frac{(x - \mu_k)^2}{2 \sigma^2}\}
\end{align*}	
\begin{align*}
	\includegraphics[scale=0.55]{Graph14}
\end{align*}

$\Rightarrow$ All densities have the \underline{same variance} but different mean $\mu_K$\\
Plugging that into (\ref*{Eq. 4}), we get
\begin{align}
	P_K(x) = \frac{\pi_K \exp\{-1/2\sigma^2 (x - \mu_K)^2\}}{\sum\limits_{l = 1}^K \pi_l \exp\{-\frac{1}{2\sigma^2}(x - \mu_K)^2\}} \label{Eq. 5}
\end{align}
$\rightsquigarrow$ Choose class $K$ for which $P_K(X)$ is largest!

\begin{itemize}
	\item We can forget about the denominator of (\ref*{Eq. 5}): He's the same number for each class $K = 1, \ldots, K$
	\item Hence pick class that maximizes $\pi_K \exp \{- \frac{1}{2 \sigma^2} (x - \mu_K)^2\}$
	\item \underline{Equivalently}: Pick class that maximizes \begin{align*}
		\log \pi_K - \frac{(x - \mu_K)^2}{2 \sigma^2} = \log \pi_K - \underbrace{\frac{x^2}{2 \sigma^2}}_{\text{does not depend on $K$}} + \frac{x \mu_K}{\sigma^2} - \frac{\mu_K^2}{2 \sigma^2}
	\end{align*}
	$\Rightarrow$ Pick  $K$ that maximizes: \begin{align*}
		\log \pi_K + \frac{x \mu_K}{\sigma^2} - \frac{\mu_K^2}{2 \sigma^2} \equiv \delta_K(x)
	\end{align*}
\end{itemize}

\begin{align*}
	\includegraphics[scale=0.55]{Graph15}
\end{align*}

In practice we don't know $\mu_K$ and $\sigma^2 \Rightarrow$ Use estimates (average of $X$ for observations in class $K$): \begin{eqnarray*}
	\hat{\mu}_k &=& \frac{1}{n_k} \sum\limits_{i: y_i = k} X_i\\
	\hat{\sigma}^2 &=& \frac{1}{(n-k)} \sum\limits_{k = 1}^K \sum\limits_{i: y_i = k} (X_i - \hat{\mu}_k)^2,
\end{eqnarray*}
where $n = \#$ of training data and $n_k = \#$ of training data in class $k$.
	
\section{Lecture June 13th}

see slides\\
\tit{Review: Discriminant Analysis}\\

Goal: Classify $Y \in \{1, 2, \ldots, K\}$ based on predictor $X$\\
Bayes Formula: \begin{align}
	Pr(Y = K| X = x) = \frac{\pi_k f_k(x)}{\sum\limits_{l = 1}^K \pi_l f_l(x)}, \label{Eq. 6}
\end{align}
where $\pi_k$ are prior probabilities, and $f_k(x)$ are group-specific densities for $X$.\\
\colorbox{green}{Needed}: Assumptions/estimates for $\pi_k$ and $f_k(x)$.\\
Assume that $f_k$'s are normal with class-specific mean $\mu_k$ and common variance $\sigma^2$, i.e. \begin{align*}
	f_k(x) = \frac{1}{\sqrt{2 \pi} \sigma} exp \left( - \frac{(x- \mu_k)^2}{2 \sigma^2}\right)
\end{align*}
Decision rule: Choose $k$ for which \begin{align*}
	\delta_k(x) = x \frac{\mu_k}{\sigma^2} - \frac{\mu_k^2}{2 \sigma^2} + \log \pi_k
\end{align*}
is largest. In practice: Use esimates $\hat{\mu}_k$, $\hat{\sigma}^2$

Example: Figure 4.4 in the book\\

\tit{LDA for $p > 1$}\\

Equation (\ref*{Eq. 6}) can also be used when $p > 1$, i.e. many predictors. In that case, $X = [X_1, \ldots, X_p]'$ is a vector.\\
Density $f_k(x)$ assumed to be multivariate normal, i.e. \begin{align*}
	f_k(x) = \frac{1}{(2\pi)^{p/2} |\Sigma|^{1/2}} \exp(-0.5(x - \mu_k)^T \Sigma^{-1} (x - \mu_k))
\end{align*}
Note: Covariance matrix $\Sigma$ is the same for all $k$! 
\newpage
\tit{Background: Multivariate normal} \begin{itemize}
	\item Mean vector $\mu$ gives mean for each element of $X = (X_1, \ldots, X_p)^T$
	\item Covariance matrix $\Sigma$ gives (co-)variances of $X$, with \begin{align*}
		\Sigma_{[I,j]} = Cov(X_I, X_j) \text{ and } \Sigma_{[I, I]} = Var(X_I) 
	\end{align*}
	\item example pictures for density, see book Figure 4.5
\end{itemize}

Classification rule: Assign to class $k$ for which \begin{align*}
	\delta_k(x) = x^T \Sigma^{-1} \mu_k - 0.5 \mu_k^T \Sigma^{-1} \mu_k + \log \pi_k
\end{align*}
is largest.\\
In practice: Use estimates $\hat{\mu}_k$ and $\hat{\Sigma}$\\

Example: Figure 4.6\\

\tit{Quadratic Discriminant Analysis}
\begin{itemize}
	\item Now assume that $X|Y = k$ is multiv. normal with mean $\mu_k$ and covariance matrix $\Sigma_k$.
	\item $\neq$ LDA, where the covariance is the same for all $k$
	\item For QDA, discriminant function $\delta_k(x)$ becomes a quadratic function of x (see eq. 4.23 in the book)
\end{itemize}
Example: Figure 4.9\\

\tit{QDA: Numerical example}\\

Consider QDA with a single predictor ($p = 1$)
\begin{itemize}
	\item Formula for $\delta_k(x)$?
	\item Consider the case of only $K = 2$ classes, and let \begin{itemize}
		\item $\pi_1 = \pi_2 = 0.5$
		\item $\mu_1 = 0$, $\sigma_1^2 = 4$
		\item $\mu_2 = 1$, $\sigma_2^2 = 1$
	\end{itemize}
	\item Sketch $f_1(x), f_2(x)$ in a single plot
	\item Which values for $x$ get assigned to class 1?
\end{itemize}

\begin{align*}
	f_k(x) = \frac{1}{\sqrt{2 \pi \sigma_k^2}} \exp \left(- \frac{(x- \mu_k)^2}{2 \sigma_k^2}\right) \id \text{(QDA)}
\end{align*}
\begin{align}
	P(Y = k| X = x) = \frac{\pi_k f_k(x)}{\sum\limits_{l = 1}^K \pi_l f_l(x)} \label{Eq. 7}
\end{align}


$\rightsquigarrow$ Choose $K$ which maximizes (\ref*{Eq. 7}), given QDA assumptions\\
maximize $\pi_k f_k(x)$ \begin{align*}
	= \frac{\pi_k}{\sigma_k \sqrt{2 \pi}} \exp \left(- \frac{(x- \mu_k)^2}{2 \sigma_k^2}\right) = \log \pi_k - \log \sigma_k - 1/2 \log 2\pi - \frac{(x-\mu_k)^2}{2 \sigma_k^2}
\end{align*}
$\rightsquigarrow$ Choose $K$ that maximizes \begin{align*}
	\delta_k(x) = \log \pi_k - \log \sigma_k - \frac{(x - \mu_k)^2}{2 \sigma_k^2}
\end{align*}
Now let $K = 2$, $\pi_1 = \pi_2 = 0.5$, $\mu_1 = 0, \mu_2= 1$, $\sigma_1^2 = 4, \sigma_2^2 = 1$\\

$\delta_1 (x) = \log(1/2) - \log(2) - \frac{(x-0)^2}{8} = - 2 \log(2) - \frac{x^2}{8}$\\
$\delta_2 (x) = - \log(2) - \log(1) - \frac{(x-1)^2}{8} = -\log(2) - \frac{(x-1)^2}{8} $\\

Classify to $K = 1$ iff
\begin{eqnarray*}
	\delta_1(x) &>& \delta_2(x)\\ \Leftrightarrow \delta_1(x) - \delta_2(x) &>& 0\\ \Leftrightarrow - \log(2) - \frac{x^2}{8} + \frac{(x-1)^2}{2} &>& 0\\
	- \log(2) - \frac{1}{8} x^2 + \frac{x^2}{2} - \frac{2x}{2} + \frac{1}{2} &>& 0\\
	(- \frac{1}{8} + \frac{4}{8}) x^2 - x + \frac{1}{2} - \log(2) &>& 0
\end{eqnarray*}  
$\Rightarrow$ Find $x^*$ such that $\delta_1(x^*) - \delta_2(x^*) = 0$ 
\begin{align*}
	x^* = \frac{1 \pm \sqrt{1 - \frac{3}{2} \cdot c}}{\frac{3}{4}} = \{-0.181; 2.847\} 
\end{align*}

\tit{QDA when $p$ is large}
\begin{itemize}
	\item One covariance matrix contains \begin{align*}
		0.5p (p+1)
	\end{align*}
	parameters
	\item Need one for each class...
	\item Hence, large training data set needed
\end{itemize}

\tit{Comparison of classification methods} 
\begin{itemize}
	\item LDA and logistic regression are quite similar \begin{itemize}
		\item Both have linear decision boundaries, see p. 151 in book
		\item Slightly different assumptions, reflected in estimation
	\end{itemize}
	\item Flexibility: QDA beats LDA
	\item Stability: LDA beats QDA
	\item KNN can be good, but $K$ must be chosen with care \begin{itemize}
		\item Bias-variance trade-off
		\item Cross-validation methods
	\end{itemize}
\end{itemize}

\tit{Simulation example in the book (p.152)}\\
Six scenarios; $p = 2$ predictors in each. \begin{itemize}
	\item Scenarios 1/2: $X$ follows a multivariate normal (as in LDA assumptions)
\end{itemize}

\section{Lecture June 14th} 

False positive rate = $\frac{\# \{\text{Prediction = Yes, Outcome = No}\}}{\# \{\text{Outcome = No}\}} =$ Probability of predicting 'Yes', given that outcome is 'No'\\

True positive rate = $\frac{\# \{\text{Prediction = Yes, Outcome = Yes}\}}{\# \{\text{Outcome = Yes}\}} =$ Probability of predicting 'Yes', given that outcome is 'Yes'.\\

\tit{ROC curve:}\\
\begin{align*}
	\includegraphics[scale=0.55]{Graph16}
\end{align*}

A: Always predict 'No' $\rightsquigarrow$ no false positives, no true positives\\
B: Always predict 'Yes' $\rightsquigarrow$ FP rate = TP rate = 1\\
$\rightsquigarrow$ Good predictor attains good tradeoff $\rightsquigarrow$ Area between ROC curve and $45^{\circ}$-line is large\\
\newpage
\underline{Note:} \begin{itemize}
	\item Statistical model (like LDA or logistic regression) gives a set of probabilities $\{p_i\}_{i \in I}$, $I = \{1, \ldots, n\}$, where \begin{align*}
		\hat{p}_i = (\text{$\widehat{Prob}$ obs. $i$ = Yes})
	\end{align*}
	\item Classification/prediction rule: Classify observation $i$ as 'Yes' if $\hat{p}_i > \tau$
	\item For each choice of $\tau$, we get a certain point on the ROC curve (i.e. FP rate, TP rate)
	\item If the model is very good, each choice of $\tau$ leads to some point 'far away' from the $45^{\circ}$-line\\
	(except that $\tau = 0$ always leads to point B, and $\tau = 1$ always leads to point A)
\end{itemize}

\paragraph{Resampling Methods (Chapter 5)}
$ $\\

Broad idea: \begin{itemize}
	\item Draw repeated samples from training data
	\item Fit a given statistical model in each iteration
	\item Examine model fit
\end{itemize}

Main purpose: Estimate a model's test sample performance (cross-validation)\\

\paragraph{Cross validation (5.1)}
$ $\\

Question: How to estimate the test-sample performance of a given model (say, $\hat{f}$)?
\begin{itemize}
	\item Simply using the training sample performance of $\hat{f}$ as a proxy for the test sample performance is a bad idea (the two may differ a lot, e.g. due to overfitting)
	\item Some methods (e.g. adjusted $R^2$) make a mathematical adjustment to training sample performance in order to estimate test sample performance $\rightsquigarrow$ see chapter 6 in book 
	\item Here: Discuss cross-validation as an estimator for test sample performance
\end{itemize}

\paragraph{Validation set approach (5.1.1)}
$ $\\

Randomly split sample in two parts (training and validation parts). Fit the model to the training part, and see how it performs on the validation sample.\\
$\id \id \rightsquigarrow$ see figure 5.1 in the book\\
\tit{Pro:} Easy to understand and to implement\\
\tit{Contra:} You obtain a pessimistic estimate of the model performance (you use only n/2 rather than n observations for model fitting). The Estimate of the test error is quite variable, depends on random split into training vs. validation sample.\\

\paragraph{Leave-one-out Cross-validation (5.1.2)}
$ $\\

\tit{Idea:} Fit model to observations $i = 2, \ldots, n$, then compute $MSE_1 = (Y_1 - \hat{Y}_1)^2$. Then use observations $i = 1, 3, 4, \ldots,n$, and compute $MSE_2 = (Y_2 - \hat{Y}_2)^2$\\

Repeating this $n$ times gives $n$ squared errors, $MSE_1, \ldots, MSE_n$.\\
The LOOCV estimate of the test error is then given by \begin{align*}
	CV_{(n)} = \frac{1}{n} \sum\limits_{i = 1}^n MSE_i
\end{align*}
\tit{Pro:} (Compared to validation set approach) \begin{itemize}
	\item Get realistic estimate of test error (use $n-1 \approx n$ observations for model fitting)
	\item No randomness in training/validation splits
\end{itemize}

\tit{Con:} \begin{itemize}
	\item Need to fit model n times $\rightsquigarrow$ may take very long time 
\end{itemize}

\section{Lecture June 20th} 

Cross-Validation: Tool to estimate a model's performance for new (AKA test) data\\

\paragraph{$K$-fold Cross-Validation (5.1.3)}
$ $\\

Idea: Divide data into $K$ groups (or folds) of approximately equal size. 
Then use the first group as a validation set, and groups 2 to K for model fitting. Compute MSE as the average squared error in group 1.\\
Repeat this process $K$ times, using a different group as validation set each time. The $K$-fold Cross Validation estimate is then given by \begin{align*}
	CV_{(K)} = \frac{1}{K} \sum\limits_{i = 1}^K MSE_i
\end{align*}
Note that LOOCV (Leave-One-Out-Cross-Validation) is a special case of $K$-fold Cross Validation, with $K = n$. In practice, one often uses $K = 5$ or $K = 10$.

\tit{Pro}: \begin{itemize}
	\item Much faster than LOOCV (need to refit model 5 or 10 times instead of $n$ times)
	\item Estimate of the test error has a lower variance than for LOOCV (see below).
\end{itemize}

\tit{Example}: Want to compare linear regression to a more fancy method. \begin{itemize}
	\item Run $K$-fold CV for linear regression, get estimate $CV_{(K)}^{LR}$ (LR = linear regression)
	\item Do the same for the fancy method, get $CV_{(K)}^{FM}$\\
	$\rightsquigarrow$ Choose method with better (=lower) $CV_{(K)}$ estimate. 
\end{itemize}
\newpage
\paragraph{Bias-Variance Trade-Off for $K$-fold Cross-Validation (5.1.4)}
$ $\\
\begin{itemize}
	\item LOOCV gives nearly unbiased estimate of the test MSE (use $n-1 \approx n$ observations for the model fitting)
	\item With $K$-fold CV, we use $\frac{K - 1}{K} \times n$ observations for the model fitting\\
	$\rightsquigarrow$ introduces some bias since we are using fewer than $n$ observations for fitting. 
	\item In LOOCV, we average over $n$ training samples that are almost identical, hence the MSE estimates are highly correlated. In $K$-fold Cross Validation, we average over $K$ training samples that have less overlap, and are thus less correlated. 
\end{itemize}
$\rightsquigarrow$ To sum up, $K$-fold CV gives an estimator of test MSE that has (i) more bias and (ii) less variance than the estimator from LOOCV.\\
In practice, $K = 5$ or $K = 10$ often work well.\\

\tit{Note}: For logistic regression, one can replace the MSE criterion by the log-likelihood function. One could also use the MSE for probabilities, though: If $\hat{p}_j$ is the predicted probability for observation j, and observation $Y_j \in \{0,1\}$, then \begin{align*}
	(Y_j - \hat{p}_j)^2 = \begin{cases}
	(1 - \hat{p}_j)^2 \id \text{ if } Y_j = 1\\
	\hat{p}_j^2 \id \id \id \text{ if } Y_j = 0
	\end{cases}\\
\end{align*}

\section{Lecture June 21st}

\paragraph{Chapter 6: Linear model selection and regularization}
$ $\\

Setup: Regression model \begin{align}
	Y = \beta_0 + \beta_1 X_1 + \ldots + \beta_p X_p + \epsilon \label{Eq. 8}
\end{align}
(e.g. $Y$ = Goal difference between Team 1 and Team 2, $X_1, \ldots, X_p$ encode information on both teams, such as performance, betting odds, info on players, etc.)\\
$\rightsquigarrow$ $p$ can be quite large \begin{itemize}
	\item often unclear ex ante which variables are informative
	\item considering interactions and polynomials further increases $p$
\end{itemize}

Until now: Fitted Equation (\ref*{Eq. 8}) with OLS. This chapter: Look at fitting procedures that are better in terms of \underline{prediction accuracy} and/or \underline{interpretability}.
\begin{itemize}
	\item \underline{Prediction accuracy}: If there are many predictors ($p$ is large), OLS prediction has large variance.\\
	Solution: Decrease variance by $\underbrace{\text{constraining}}_{\text{e.g. set to zero}}$ or $\underbrace{\text{shrinking}}_{\text{e.g. shrink toward zero}}$ some of the coefficient estimates $\hat{\beta}_j$
	\item \underline{Interpretability}: In many data sets, coefficients $\beta_1, \ldots, \beta_p$ contain irrelevant variables that are not associated with $Y$. The presence of many variables $j$ with $\beta_j$ 'close to zero' (but not exactly zero) makes OLS hard to interpret.\\
	Solution: Variable methods for choosing a relevant subset of predictors.\\ 
\end{itemize}

\paragraph{Subset Selection (6.1)}
$ $\\

Idea: Find the best subgroup of all $p$ possible vectors.\\ 
\newpage
\tit{Algorithm 6.1(from the book): Best Subset Selection} \begin{enumerate}
	\item Let $M_0$ = null model (no predictors, i.e. $\hat{Y} = \overline{Y}$ (training sample average))
	\item For $k = 1, 2, \ldots, P$\\
	(a) Fit all $\binom{P}{k}$ models that contain exactly $k$ predictors\\
	(b) Let $M_k$ be the best of theses models (in terms of $R^2$) 
	\item Select a simple best model from among $M_0, \ldots, M_P$ using CV (or adjusted $R^2$)\\
\end{enumerate}

Note: Adjusted $R^2 = 1 - \frac{RSS/(n-d-1)}{TSS/(n-1)}$,\\
where $RSS = \sum\limits_{i = 1}^n (Y_i - \hat{Y}_i)^2$, $TSS = \sum\limits_{i = 1}^n (Y_i - \overline{Y})^2$, $d = \#$ of predictors\\
$\rightsquigarrow$ Like $R^2$, but imposes a penalty on the $\#$ of predictors, d\\
$\rightsquigarrow$ Fixes the tendancy of $R^2$ to select overfitted models\\
$\rightsquigarrow$ Like CV, adjusted $R^2$ is an estimate of a model's test error (adjusted $R^2$ simpler, CV tends to be more accurate)

\section{Lecture June 28th} 

\paragraph{Stepwise selection (6.1.2)}
$ $\\

Best Subset Selection runs into problems when $p$ gets large:\\
Need to evaluate many (!) combinations of variables $\rightsquigarrow$ very expensive in terms of computer time\\
$\rightsquigarrow$ One fix: Forward step-wise selection\\ 

\tit{Algorithm 6.2 (from the book): Forward stepwise selection} 

\begin{enumerate}
	\item Let $M_0$ denote the null model without any predictors
	\item For $k = 0, \ldots, p-1$ \begin{enumerate}
		\item Consider all $p-k$ models that add \underline{one} additional predictor to $M_K$
		\item Choose the best among these $p-k$ models, and call it $M_{k+1}$ (use $R^2$ as criterion)
	\end{enumerate}
	\item Select a single best model from $M_0, \ldots, M_p$ using CV or adjusted $R^2$ (or BIC, Akaike,... )
\end{enumerate}

\underline{Note}: This is much simpler than Best Subset Selection: At step $K$, we need to consider only $p-k \ll \binom{p}{k}$ models.\\

Alternative Strategy to deal with many predictors (i.e., large $p$):\\
Shrinking the coefficients towards zero $\rightsquigarrow$ We'll look at two shrinkage methods, Ridge Regression and Lasso.\\
\newpage
\paragraph{Ridge Regression (6.2.1)}
$ $\\

OLS minimizes \begin{align*}
	RSS = \sum\limits_{i = 1}^n (Y_i - \beta_0 - \sum\limits_{j = 1}^p \beta_j X_{ij})^2
\end{align*}

Ridge Regression minimizes \begin{align*}
	 RSS + \lambda \sum_{j = 1}^p \beta_j^2,
\end{align*}
where $\lambda \geq 0$ is a tuning parameter.\\
The term $\lambda \sum\limits_{j = 1}^p \beta_j^2$ is  called a shrinkage penalty (SP)
\begin{itemize}
	\item Small when $\beta_j$'s are close to zero\\
	\rotatebox[origin=c]{180}{$\Lsh$} Hence, sets the incentive to choose 'smaller' $\beta_j$'s
	\item $\lambda = 0$, No SP, gets OLS estimates
	\item $\lambda$ large: SP grows, coefficients get smaller (as $\lambda \rightarrow \infty$, all $\beta_j$'s get to zero)
	\item Get different set of coefficients for each possible value of $\lambda$
	\item By construction, $\beta_j$'s are not exactly equal to zero (cannot easily tell which variables are 'in' or 'out')\\
	$\rightsquigarrow$ makes sense for prediction, but not for interpretation
	\item Good choice of $\lambda$ is crucial, can be done using CV
\end{itemize}
\newpage
\tit{Implementation}: 
\begin{itemize}
	\item SP does not affect the intercept, $\beta_0$\\
	\item Before running Ridge Regression, one should standardize the predictors using the following formula \begin{align*}
		\widetilde{X}_{ij} = \frac{X_{ij} - \overline{X}_j}{\sqrt{\frac{1}{n} \sum\limits_{i = 1}^n (X_{ij} - \overline{X}_j)^2}}
	\end{align*}
	with $\overline{X}_j = \frac{1}{n} \sum\limits_{i = 1}^n X_{ij}$\\
	$\rightsquigarrow \widetilde{X}_{ij}$ has mean 0 and variance 1.
\end{itemize}

Why does Ridge improve upon OLS? E.g., suppose the true model is linear, but there are many predictors.\\

\setlength\arraycolsep{25pt}
\begin{center}
	$\begin{matrix}
	\text{\underline{OLS} ($\lambda = 0$)} & \text{\underline{Ridge} ($\lambda > 0$)}\\
	\text{no bias} & \text{bias}\\
	\text{higher variance} & \text{lower variance}
	\end{matrix}$\\
\end{center}

In general, increase in $\lambda$ leads to an increase in (squared) bias, a decrease in variance and an increase \underline{or} decrease in MSE\\
\id \rotatebox[origin=c]{180}{$\Lsh$} Use CV to find good choice of $\lambda$
\begin{align*}
	\includegraphics[scale=0.55]{Graph17}
\end{align*}

Possible factors determining optimal choice of $\lambda$ \begin{itemize}
	\item If predictors are noisy, and have little explanatory power: Expect large $\lambda^*$, i.e. don't use predictors a lot, set coefficients to 'almost zero'
	\item If training sample size (n) is large: Expect small value $\lambda^*$, since estimation noise plays a smaller role
\end{itemize}

Drawback of Ridge: Includes all $p$ predictors in final model \begin{itemize}
	\item Penalty $\lambda \sum\limits_{j = 1}^p \beta_j^2$ shrinks coefficients toward zero, but not exactly equal to zero
	\item Not a problem for prediction, but for interpretation (makes it harder to explain model choice)
\end{itemize}

\rotatebox[origin=c]{180}{$\Lsh$} Seek a method that keeps benefits of Ridge (shrinkage/prediction) but is easier to interpret\\
\rotatebox[origin=c]{180}{$\Lsh$} Fix: Lasso (least absolute shrinkage and selection operator)\\

Lasso coefficients minimize \begin{align*}
	\underbrace{\sum\limits_{i = 1}^n (Y_i - \beta_0 - \sum_{j = 1}^p \beta_j X_{ij})^2}_{RSS} + \lambda \sum\limits_{j = 1}^p |\beta_j|
\end{align*}

\underline{Note}: Lasso penalty $\neq$ Ridge penalty = $\lambda \sum\limits_{j = 1}^p \beta_j^2$\\
Lasso penalty is called an $L_1$ penalty, since $L_1$ norm of $\beta = (\beta_1, \ldots, \beta_p)$ is given by $\| \beta\|_1 = \sum\limits_{j = 1}^p |\beta_j|$\\

Effect of $L_1$ penalty: When $\lambda$ is sufficently large, some of the coefficients get kicked out completely (i.e., $\beta_j = 0$ for these coefficients). Hence Lasso performs both variable selection and shrinkage, and is thus easier to interpret than Ridge.

\section{Lecture July 3rd}

Another formulation for Ridge and Lasso\\
One can show, that Ridge Regression solves the following problem: \begin{align*}
	\min\limits_{\beta} \{ \underbrace{\sum\limits_{i = 1}^n (Y_i - \beta_0 - \sum\limits_{j = 1}^p \beta_j X_{ij})^2}_{= RSS(\beta)}\} \id \text{ subject to } \id \sum\limits_{j = 1}^p \beta_j^2 \leq s
\end{align*}
Similarily, Lasso solves \begin{align*}
	\min\limits_\beta RSS(\beta) \id \text{ subject to } \id \sum\limits_{j = 1}^p |\beta_j| \leq s
\end{align*}

\tit{Notes}: \begin{itemize}
	\item $s$ is a 'budget' that we can 'spend' on the coefficients $\beta_1, \ldots, \beta_p$
	\item  $s \to \infty$: Constraint is non-binding, and we end up at the OLS solution for $\beta$
	\item $s = 0$: Get $\beta_0 \neq 0$, $\beta_1 = \ldots = \beta_p = 0$
	\item More generally: There is a one-to-one correspondence between $\lambda$ (weight on the coefficient penalty, we used before) and $s$.\\
	That means for each $s$ we can find a value of $\lambda$ that results in the same Ridge (or Lasso) coefficients. Larger values of $s$ (large budget) correspond to smaller values for $\lambda$ (small penalty).
\end{itemize}

Best Subset Selection (BSS) can also be written as a 'budget' problem: It solves \begin{align*}
	\min\limits_{\beta} RSS(\beta) \id \text{ subject to } \id \sum\limits_{j = 1}^p \mathbbm{1}(\beta_j \neq 0) \leq s
\end{align*}
In words: BSS chooses best model with at most $s$ nonzero coefficients.\\
$\rightsquigarrow$ As noted, earlier, BSS is tedious since it entails $\binom{p}{s}$ estimation problems.\\
$\rightsquigarrow$ Lasso/Ridge are feasible alternatives to BSS.\\
\newpage
\tit{Comparing Ridge and Lasso} \begin{itemize}
	\item Lasso is easier to interpret than Ridge\\
	\rotatebox[origin=c]{180}{$\Lsh$} can look at set of 'selected' variables
	\item Which one is better at prediction?\\
	Hard to tell, ultimately an empirical question. Ridge 'assumes' that there are many variables with small effects/coefficients, Lasso assumes, that there are a few key variables with big effects/coefficients (sparsity).
\end{itemize}

Figure 6.7 (book): Graphical illustration of budget formulation for Lasso (left) and Ridge (right).\\
Lasso: Budget represented as a diamond type object\\
Ridge: Budget represented as a circle\\
Coefficient solution = tangential point between budget and RSS contours (= lowest RSS value that's within the budget).\\
$\rightsquigarrow$ Figure illustrates that the Lasso budget is more likely to yield zero solutions (for $\beta_1$ in this case) 

\paragraph{High-Dimensional Data (6.4.1)}

\begin{itemize}
	\item Traditional Statistics: Large $n$, small $p$ \begin{itemize}
		\item Medicine: Predict blood pressure of $n$ patients based on age, gender, BMI 
		\item Econ: Predict wage based on a few features (experience, education, industry dummies)
	\end{itemize}
	\item 21st century: $p$ may be large \begin{itemize}
		\item Often: Much cheaper/easier to increase $p$ than increase $n$
		\item Electronic tracking (e.g. browser, fitness gadgets): No limit on $p$, what about $n$?
		\item Similarily, text mining
		\item Statistical ways to increase $p$ (interactions, polynomials)
	\end{itemize}
\end{itemize}
\newpage
\paragraph{What goes wrong in high dimensions? (6.4.2)} 
$ $\\ 

Drastic example: $p = n$ \begin{itemize}
	\item Linear model (OLS) gives percet fit
	\item I.e., $\hat{y}_i = y_i$ for all $i$ in training sample
	\item Overfitting, i.e. test sample performance likely very poor
\end{itemize}
Similar (but less drastic) phenomena if $p$ is just a little smaller than $n$

\paragraph{Regression in high dimensions (6.4.3)} 
$ $\\

Linear model $Y = \beta_0 + \beta_1X_1 + \ldots + \beta_pX_P + \epsilon$, with large $p$ \begin{itemize}
	\item Model fitting: BSS/FSS, Ridge, Lasso much better than OLS
	\item Complexity reduction via selection (BSS, Lasso) and shrinkage (Ridge, Lasso)
\end{itemize}

\tit{Prediction when $p$ is large: Key Points} \begin{itemize}
	\item Shrinkage helps!
	\item Tuning parameter selection (degree of shrinkage, $\lambda$) important
	\item Adding new predictors: Signal or noise 
	\begin{itemize}
		\item Signal helps, noise hurts
		\item Ideally: Algorithm separates signal from noise
		\item In practice: Common sense helps to design/select features
	\end{itemize}
\end{itemize}

\paragraph{Interpreting Results in High Dimensions (6.4.4)} 
$ $\\
Don't overstate results
\begin{itemize}
	\item Identify subset of useful features, even if they don't form the true model
	\item Correlation vs. causation 
\end{itemize}
Communication: Report test-sample or cross-validation metrics, not training-sample stuff ($p$-values, $R^2$)\\
Example: Report subset of features that gets selected when $\lambda = \lambda^*$ is selected by CV.

\section{Lecture July 4th} 

\paragraph{Chapter 8: Tree Based-Methods}
$ $\\ 
\begin{itemize}
	\item Regression or classification trees (Ch. 8.1) are intuitive and have natural interpretation
	\begin{itemize}
		\item Useful for interpretation/communication
		\item See examples in slides
	\end{itemize}
	\item Prediction accuracy inferior to other methods
	\item Ensemble methods (Ch. 8.2) greatly improve trees' prediction accuracy
	\begin{itemize}
		\item Tree $\Rightarrow$ Random Forests \& co.
		\item Prediction accuracy highly competitive, interpretation not straightforward
	\end{itemize}
\end{itemize}

\textit{Figure 8.1 and 8.2} 

\tit{Trees: Terminology} 
\begin{itemize}
	\item Tree is defined by binary splits of predictor space
	\begin{itemize}
		\item Recursively put observations in two bins
	\end{itemize}
	\item $R_1$, $R_2$, $R_3$ from example are terminal nodes or leaves
	\item Other (non-terminal) nodes are called internal nodes
	\begin{itemize}
		\item Branches connect several nodes (e.g. left vs right branch)
	\end{itemize}
\end{itemize}

\tit{Prediction via trees} 
\begin{itemize}
	\item Basis: Predictor space, i.e. set of possible values for $X_1, \ldots, X_p$
	\item Tree prediction: \begin{itemize}
		\item Divide predictor space into $J$ distinct and non overlapping regions, $R_1, \ldots, R_J$
		\item Prediction is the same for all observations that fall into a region $R_j$: Mean response for all observations in $R_j$
	\end{itemize}
\end{itemize}
\newpage
\tit{RSS of tree}: \begin{align*}
	\sum\limits_{j = 1}^J \underbrace{\sum\limits_{i: x_i \in R_j} (y_i - \hat{y}_{R_j})^2}_{ = \text{ RSS within leaf } R_j}, 
\end{align*}
$\hat{y}_{R_j}$ = prediction for observation in leaf $R_j$ (same for all $i: x_i \in R_j$)\\

\tit{How to find the partitions $R_1, \ldots, R_J$} 

Goal: Find partitions which minimize RSS, \begin{align*}
	\sum\limits_{j = 1}^J \sum\limits_{i: x_i \in R_j} (y_i - \hat{y}_{R_j})^2
\end{align*}
Note: prediction is the same for all $i \in R_j$!

\underline{Given} choice of leaves $R_1, \ldots, R_J$: optimal to set \begin{align*}
	\hat{y}_{R_j} = \frac{1}{|R_j|} \sum\limits_{i: x_i \in R_j} y_i, = \text{ mean of  $y_i$ within leaf $R_j$}
\end{align*}
$|R_j|$ = number of observations within leaf $R_j$

\tit{Recursive binary splits}

\begin{itemize}
	\item Find, best split given all previous splits
	\item 'Greedy', i.e. don't incorporate possible future splits
	\item Procedure: Find variable $j$ and split point $s$ which minimize \begin{align*}
		RSS(j,s) = \sum\limits_{i:x_i \in R_1(j,s)} (y_i - \hat{y}_{R_1})^2 + \sum\limits_{i: x_i \in R_2(j,s)} (y_i - \hat{y}_{R_2})^2
	\end{align*}
	where $R_1(j,s)$ and $R_2(j,s)$ are the two 'buckets' that result from using $j$ and $s$ for splittiing: \begin{align*}
		R_1 = \{X|X_j < s\}, \id R_2 = \{X|X_j \geq s\}
	\end{align*}
\end{itemize}

Splitting always reduces RSS\\
\newpage
(very) simple example: Only one predictor ($p = 1$), two observations ($n = 2$)
\begin{center}
	$\begin{matrix}
		i & x_i & y_i\\
		\hline
		1 & 0 & 2\\
		2 & 4 & 5
	\end{matrix}$
\end{center}
\underline{Before} split: RSS = $(2 - 3.5)^2 + (5 - 3.5)^2 = 2 \cdot 2.25$\\
\underline{Split}: Only have one prediction ($X$) that can be used for splitting. all 'reasonable split points are $\in (0,4)$ e.g. split at $\{X < 2\}$ vs. $\{X \geq 2\}$\\
\id $\rightsquigarrow$ each bucket contains only one observation, RSS = 0\\

Example shows that splitting always reduces RSS, up to a (crazy) limiting case where each bucket contains exactly one observations\\

\tit{Gini index} \begin{align*}
	G = \hat{p}_m (1- \hat{p}_m)
\end{align*} \begin{align*}
	\includegraphics[scale=0.55]{Graph18}
\end{align*}
Idea: Want to have buckets that separate 'zeros' from 'ones' (i.e. sharp prediction)
\newpage
\tit{Finding the optimal tree size}

\begin{itemize}
	\item Size $|T|$ of a tree $T$ can be measured by the number of terminal nodes (three in simple example above)
	\item Cost-complexity pruning: Grow very large tree $T_0$, then find optimal subtree $T \subset T_0$ that minimizes \begin{align*}
		\sum\limits_{m = 1}^{|T|} \sum\limits_{i: x_i \in R_m} (y_i - \hat{y}_{R_m})^2 + \alpha |T|
	\end{align*}
	where $\alpha$ is a tuning parameter that penalizes complexity 
	\item $\alpha$ can be determined via cross-validation
\end{itemize}

\tit{Algorithm: (8.1 in the book)} 
\begin{enumerate}
	\item Use recusive binary splitting to grow a large tree on the training data, stopping only when each terminal node has fewer than some minimum number of observations
	\item Apply cost complexity pruning to the large tree in order to obtain a sequence of the best subtrees, as a function of $\alpha$
	\item Use $K$-fold cross-validation to choose $\alpha$. That is, divide the training observations into $K$ folds. For each $k = 1, \ldots, K$: \begin{enumerate}
		\item Repeat Steps 1 and 2 on all but the $k$th fold of the training data
		\item Evaluate the mean squared prediction error on the data in the left-out $k$th fold, as a function of $\alpha$.
	\end{enumerate}
	Average the results for each value of $\alpha$, and pick $\alpha$ to minimize the average error
	\item Return the subtree from Step 2 that corresponds to the chosen value of $\alpha$
\end{enumerate}

\tit{Large tree $T_0$ for the Baseball data}\\
\textit{Figure 8.4}\\
\tit{MSE plotted against tree size}\\
\textit{Figure 8.5}

\section{Lecture July 5th} 

\tit{Classification Trees} 
\begin{itemize}
	\item Like regression tree, but for a qualitative varibale
	\item Main idea is the same 
	\item Focus on binary variable (two classes) here
	\item Different criteria for making binary splits: \begin{itemize}
		\item Gini index \begin{align*}
			G = 2 \hat{p}_m (1 - \hat{p}_m)
		\end{align*}
		where $m$ is the region (or bucket) of interest, and $\hat{p}_m$ is the fraction of ones
		\item Idea: Good split (small $G$) contains only zeroes or ones
		\item Alternatively: Cross-entropy, similar idea
	\end{itemize}
\end{itemize}

\tit{Classification Tree: Heart Disease Data}\\
\textit{Figure 8.6} 

\tit{Criteria for making binary splits}
\begin{itemize}
	\item Theoretically: Aim to minimize missclassification rate \begin{align*}
		E = 1 - \max(\hat{p}_m, 1 - \hat{p}_m)
	\end{align*}
	However, does not work well in practice
	\item Smoother alternatives: \begin{itemize}
		\item Gini index, \begin{align*}
		G = 2 \hat{p}_m(1 - \hat{p}_m),
		\end{align*}
		where $m$ is the region (or bucket) of interest, and $\hat{p}_m$ is the fraction of ones
		\item Cross-entropy \begin{align*}
			D = -(\hat{p}_m \log(\hat{p}_m) + (1 - \hat{p}_m) \log(1 - \hat{p}_m))
		\end{align*}
	\end{itemize}
	\item Idea: Ideal split (small $E$, $G$ or $D$) contains only zeros or ones
\end{itemize}

\tit{Trees: Pros} \begin{itemize}
	\item Easy to explain
	\item Mirror human decision making (?)
	\item Easy to draw
	\item Can easily handle qualitative predictors w/o dummies
\end{itemize}

\tit{Trees: Cons} \begin{itemize}
	\item Prediction performance not great
	\item Can be very non-robust (aka high variance)
\end{itemize}

\paragraph{Bagging, Random Forests, Boosting (8.2)} 

\begin{itemize}
	\item Until now: Trees are nice, but too shaky to forecast well
	\item Here: Ways to fix this problem
	\item Will improve prediction performance, at the cost of interpretation
\end{itemize}

\tit{Bagging regression trees}\\

\underline{Want}: Prediction at a new (test sample) point $x$, say $\hat{f}(x)$\\
If $\hat{f}(x)$ is produced by a single tree, then prediction tends to be shaky\\
\id \rotatebox[origin=c]{180}{$\Lsh$} Idea: Compute $B$ predictions from $B$ different trees, and then take the average.\\
\id $\rightsquigarrow$ Question: How to get 'different' trees?\\
\id $\rightsquigarrow$ Answer: Take $B$ random samples from training data, and fit new tree each time\\
Prediction then given by \begin{align*}
	\hat{f}_{avg} (x) = \frac{1}{B} \sum_{b = 1}^B \hat{f}^b(x),
\end{align*}
where $\hat{f}^b$ is the tree obtained from the $b$-th sample.\\
\newpage
\tit{Bagging (from the slides)} 
\begin{itemize}
	\item Stats 1: If random variables $Z_1, \ldots, Z_n$ are indepedent and $V(Z_i) = \sigma^2$, then $\frac{1}{n} \sum\limits_{i = 1}^n Z_i$ has variance $\sigma^2/n$
	\item Hence, averaging reduces variance
	\item This intuition suggests to 'somehow' combine many trees to reduce variance
	\item Ideally: Compute tree predictions $\hat{f}^1(x), \ldots, \hat{f}^B(x)$ from $B$ different training data sets,then compute low-variance prediction \begin{align*}
		\hat{f}_{avg}(x) = \frac{1}{B} \sum\limits_{b = 1}^B \hat{f}^b(x)
	\end{align*}
	\item In practice: Only a single training data set
	\item Pragmatic alternative: Bagging (Bootstrap Aggregating) 
	\item Bootstrap - What's that? 
\end{itemize}

\tit{Bootstrap: Recipe} 
$ $\\

For data set of size $n$: \begin{itemize}
	\item Draw $n$ numbers (integers) between 1 and $n$, with replacement 
	\item Get Bootstrap data set by selecting the rows (observations) drawn in the previous step
	\item Produces synthetic data set, with some of the original observations appearing more than once, others not at all
\end{itemize}

\tit{Bootstrap: Illustration}\\

\textit{Figure 5.11}\\
\newpage
\tit{Bagging: Recipe} \begin{itemize}
	\item Create $B$ bootstrap resamples of the data
	\item $\hat{f}^{*b}(x)$ = prediction for the $b$th sample
	\item Bagging prediction given by \begin{align*}
	\hat{f}_{bag}(x) = \frac{1}{B} \sum\limits_{b = 1}^B \hat{f}^{*b}(x)
	\end{align*}
	\item Simple and effective way to reduce variance of prediction
\end{itemize}

\tit{Bagging: Notes} \begin{itemize}
	\item Can be applied to any procedure $\hat{f}$, but particularly useful for trees
	\item Why? Trees are very shaky on their own
	\item Recipe for bagging trees: \begin{itemize}
		\item Each tree $b = 1, \ldots, B$ is grown very large \begin{itemize}
			\item Hence low bias, high variance
			\item No pruning, hence no choice for tuning parameter needed
		\end{itemize}
	\item Then average over $B$ trees to reduce variance
	\item Prediction \begin{itemize}
		\item Regression: See $\hat{f}_{bag}(x)$ above
		\item Classification: Do majority vote among $B$ trees
	\end{itemize}
	\end{itemize}
\end{itemize}

\tit{Excercise 5.4/2}\\

(a) Prob(1st bootstrap draw $\neq$ $j$th obs.) = 1 - Prob(1st bootstrap draw = $j$th obs.) = $\frac{n-1}{n}$\\
(b) $\frac{n-1}{n}$\\
(c) Prob($j$th obs. not in bootstrap sample)\\
\id = Prob(1st bootstrap draw $\neq j$, 2nd bootstrap draw $\neq j, \ldots,$ $n$th bootstrap draw $\neq j$)\\
\id = $(\frac{n-1}{n})^n = (1 - \frac{1}{n})^n$
\id $\Rightarrow$ Prob(obs. j is in bootstrap sample) = $1 - (1 - \frac{1}{n})^n$\\
For $n \to \infty$, $1 - (1 - \frac{1}{n})^n \to 0.632\ldots$\\

\underline{Note}: Convergence is quite fast, so that approximation works well even for a 'small' $n \approx 50$\\
$\rightsquigarrow$ Each observation has a chance of $\approx \frac{2}{3}$ of ending up in the bootstrap sample.\\

\tit{Exercise 8.4/4}
  \tikzset{
	decision/.style={diamond, minimum height=10pt, minimum width=10pt, inner sep=1pt},
	chance/.style={circle, minimum width=10pt, draw=blue, fill=none, thick, inner sep=0pt},
}

(a)  \hspace*{35mm}  (b)\\

\begin{forest}
	label L/.style={
		edge label={node[midway,left,font=\scriptsize]{#1}}
	},
	label R/.style={
		edge label={node[midway,right,font=\scriptsize]{#1}}
	},
	for tree={
		child anchor=north,
		for descendants={
			{edge=->}
		}
	},
	[$X_1< 1$, decision, draw
	[$X_2 < 1$, decision, draw, label L=Yes,
	[$X_1 < 0$, decision, draw, label L = Yes,
	[3, rectangle, draw, label L= Yes] 
	[$X_2 <0$, decision, draw, label R = No,
	[10, rectangle, draw, label L = Yes]
	[0, rectangle, draw, label R = No]
	]
	]
	[15, rectangle, draw, label R=No]
	]
	[5, rectangle, draw, label R=No]
	]
\end{forest}
\includegraphics[scale=0.55]{Graph19}

\section{Lecture July 11th}

Figure 8.8(book): Shows the impact of $B$ (\# of bagging draws) on test sample performance. \begin{itemize}
	\item Bagging average $\hat{f}_{bag}(x) = \frac{1}{B} \sum\limits_{b = 1}^B \hat{f}^{*b}(x)$ can be seen as expected value of $\hat{f}^{*b}(x)$, where the expectation is taken with respect to 'bootstrap randomness' 
	\item Typically find, that $\hat{f}_{bag}(x)$ (and thus its test sample performance) stabilize at some large value for $B$ $\rightsquigarrow$ Estimate of the expected value (see last point) does not improve much, if $B$ is further increased. 
	\item  $B$ cannot be chosen too large (if computer time was irrelevant, we would always pick a very large value of $B$) 
	\item In practice: Choose a value of $B$ that's feasible (regarding computer time) and leads to robust estimates.\\
	$\rightsquigarrow$ e.g. plot $\hat{f}_{bag}(x)$ against $B$, and check if the estimate stabilizes.\\
	$\rightsquigarrow$ If computations are fast enough, may be possible to simply choose a very large value of $B$, e.g. $B = 1000$ 
\end{itemize}

\tit{Out-of-Bag error estimation}\\

How to estimate test error for bagging-based forecast? \begin{itemize}
	\item Each bootstrap sample contains $\approx 2/3$ of the obs, $\approx 1/3$ not present
	\item Consider predicting observation $i = 1$ \begin{itemize}
		\item Select bagging iterations $b$ for which $i = 1$ is not used for model fitting (about $B/3$)
		\item  Average their prediction for $i =1$, compare to true outcome
	\end{itemize}
	\item Repeat the above for $i = 1, \ldots, n$ \begin{itemize}
		\item Valid estimate of test error (based on out-of-sample-predictions!)
	\item Very similar (but much simpler than) CV for large $B$
	\end{itemize}
\end{itemize}
\newpage
\tit{Variable Importance Measures}\\
\begin{itemize}
	\item Bagging is harder to interpret than a single tree
	\item Rough interpretation device: variable importance
	\item Suppose we want to measure the importance of the variable 'years' in the Baseball data \begin{itemize}
		\item For all $B$ trees, add the RSS reduction that are due  to splits in 'years'
	\end{itemize}
\item Similarily, add reductions in Gini measures in case of classification trees
\end{itemize}
\textit{Figure 8.9}\\

\tit{Random forests}

\begin{itemize}
	\item Small tweak of bagged trees, aiming to decorrelate the $B$ trees
	\item Each time a split is considered, select subset of $m$ predictors as split candidates \begin{itemize}
		\item Fresh sample of $m$ predictors taken at each split
		\item Popular choice is $m = \sqrt{p}$
	\end{itemize}
	\item Idea: Make $B$ trees less similar \begin{itemize}
		\item E.g., consider data set with one dominant predictor
		\item Bagging: This predictor is present in most splits
		\item  RFs: $(p-m)/p$ of the splits do no even consider the dominant predictor
	\end{itemize}
\end{itemize}

Tuning parameter $m$ (\# of split candidates) \begin{itemize}
	\item $m = p$ leads to bagged trees
	\item $m = \sqrt{p}$ often used in practice
	\item Small $m$ helpful with many correlated predictors
	\item Example: Cancer prediction \begin{itemize}
		\item $n = 349$ patients, qualitative outcome (health condition, 15 categories)
		\item $p = 500$ predictors
		\item Results on the next slide (slides Lecture 23 (slide 12), Figure 8.10)  
	\end{itemize}
\end{itemize}

\tit{Boosting}\\

\tit{Algorithm 8.2}\\
\begin{enumerate}
	\item Set $\hat{f}(x) = 0$ and $r_i = y_i$ for all $i$ in the training set
	\item For $b = 1, 2, \ldots, B$, repeat: \begin{enumerate}
		\item Fit a tree $\hat{f}^b$ with $d$ splits ($d+1$ terminal nodes) to the training data $(X,r)$\\
		\item Update $\hat{f}$ by adding in a shrunken version of the new tree: \begin{align*}
			\hat{f}(x) \leftarrow \hat{f}(x) + \lambda \hat{f}^b(x)
		\end{align*}
		\item Update the residuals, \begin{align*}
			r_i \leftarrow r_i - \lambda \hat{f}^b(x_i)
		\end{align*}
	\end{enumerate}
	\item Output the boosted model, \begin{align*}
		\hat{f}(x) = \sum\limits_{b = 1}^B \lambda \hat{f}^b(x)
	\end{align*}
\end{enumerate}

\tit{Boosting: Tuning Parameters}

\begin{itemize}
	\item Number of trees, $B$ \begin{itemize}
		\item Unlike Bagging: Can overfit if $B$ too large
		\item Unlike Bagging: Use CV to select $B$
	\end{itemize}
	\item Shrinkage parameter, $\lambda$ \begin{itemize}
		\item Controls size of updates
		\item Typical values are $\lambda = 0.01$ or $0.001$
		\item If $\lambda$ ist too small, large $B$ is needed for good performance
	\end{itemize}
	\item Size of each individual tree, $d$ \begin{itemize}
		\item Often $d = 1$ (single tree, i.e. 'stump') works well
	\end{itemize}
\end{itemize}

\tit{Bagging vs. Boosting}\\

\setlength\arraycolsep{25pt}
\renewcommand{\arraystretch}{1.5}
\begin{center}
	$\begin{matrix}
	\text{Bagging/RF} & \text{Boosting}\\
	\hline
	\text{Ensemble method (combine $B$ trees)} & \text{Ensemble method (combine $B$ trees)}\\
	\text{Individual trees are large} & \text{Individual trees are small}\\
	\text{Trees independent (Bootstrap)} & \text{Trees dependent (grown sequentially)}\\
	\text{Tuning parameter $m$ (RF only)} & \text{Tuning parameters $B$, $\lambda$, $d$}\\
	\text{No overfitting if $B$ is large} & \text{Overfitting if $B$ is large} 
	\end{matrix}$
\end{center}


\textit{Cancer Data: Figure 8.11}




\end{document}

